\documentclass{article}
\usepackage{ifluatex}
\ifluatex 
    \usepackage{fontspec}
    \setsansfont{CMU Sans Serif}%{Arial}
    \setmainfont{CMU Serif}%{Times New Roman}
    \setmonofont{CMU Typewriter Text}%{Consolas}
    \defaultfontfeatures{Ligatures={TeX}}
\else
    \usepackage[T2A]{fontenc}
    \usepackage[utf8]{inputenc}
\fi
\usepackage[english,russian]{babel}
\usepackage{amssymb,latexsym,amsmath,amscd,mathtools,wasysym}
\usepackage[shortlabels]{enumitem}
\usepackage[makeroom]{cancel}
\usepackage{graphicx}
\usepackage{geometry}
\usepackage{verbatim}
\usepackage{fvextra}

\usepackage{longtable}
\usepackage{multirow}
\usepackage{multicol}
\usepackage{tabu}
\usepackage{arydshln} % \hdashline and :

\usepackage{float}
\makeatletter
\g@addto@macro\@floatboxreset\centering
\makeatother
\usepackage{caption}
\usepackage{csquotes}
\usepackage[bb=dsserif]{mathalpha}
\usepackage[normalem]{ulem}

\usepackage[e]{esvect}
\let\vec\vv

\usepackage{xcolor}
\colorlet{darkgreen}{black!25!blue!50!green}


%% Here f*cking with mathabx
\DeclareFontFamily{U}{matha}{\hyphenchar\font45}
\DeclareFontShape{U}{matha}{m}{n}{
    <5> <6> <7> <8> <9> <10> gen * matha
    <10.95> matha10 <12> <14.4> <17.28> <20.74> <24.88> matha12
}{}
\DeclareSymbolFont{matha}{U}{matha}{m}{n}
\DeclareFontFamily{U}{mathb}{\hyphenchar\font45}
\DeclareFontShape{U}{mathb}{m}{n}{
    <5> <6> <7> <8> <9> <10> gen * mathb
    <10.95> matha10 <12> <14.4> <17.28> <20.74> <24.88> mathb12
}{}
\DeclareSymbolFont{mathb}{U}{mathb}{m}{n}

\DeclareMathSymbol{\defeq}{\mathrel}{mathb}{"15}
\DeclareMathSymbol{\eqdef}{\mathrel}{mathb}{"16}


\usepackage{trimclip}
\DeclareMathOperator{\updownarrows}{\clipbox{0pt 0pt 4.175pt 0pt}{$\upuparrows$}\hspace{-.825px}\clipbox{0pt 0pt 4.175pt 0pt}{$\downdownarrows$}}
\DeclareMathOperator{\downuparrows}{\clipbox{0pt 0pt 4.175pt 0pt}{$\downdownarrows$}\hspace{-.825px}\clipbox{0pt 0pt 4.175pt 0pt}{$\upuparrows$}}

\makeatletter
\providecommand*\deletecounter[1]{%
    \expandafter\let\csname c@#1\endcsname\@undefined}
\makeatother


\usepackage{hyperref}
\hypersetup{
    %hidelinks,
    colorlinks=true,
    linkcolor=darkgreen,
    urlcolor=blue,
    breaklinks=true,
}

\usepackage{pgf}
\usepackage{pgfplots}
\pgfplotsset{compat=newest}
\usepackage{tikz,tikz-3dplot}
\usepackage{tkz-euclide}
\usetikzlibrary{calc,automata,patterns,angles,quotes,backgrounds,shapes.geometric,trees,positioning,decorations.pathreplacing}
\pgfkeys{/pgf/plot/gnuplot call={T: && cd TeX && gnuplot}}
\usepgfplotslibrary{fillbetween,polar}
\ifluatex
\usetikzlibrary{graphs,graphs.standard,graphdrawing,quotes,babel}
\usegdlibrary{layered,trees,circular,force}
\else
\errmessage{Run with LuaTeX, if you want to use gdlibraries}
\fi
\makeatletter
\newcommand\currentnode{\the\tikz@lastxsaved,\the\tikz@lastysaved}
\makeatother

%\usepgfplotslibrary{external} 
%\tikzexternalize

\makeatletter
\newcommand*\circled[2][1.0]{\tikz[baseline=(char.base)]{
        \node[shape=circle, draw, inner sep=2pt,
        minimum height={\f@size*#1},] (char) {#2};}}
\makeatother

\newcommand{\existence}{{\circled{$\exists$}}}
\newcommand{\uniqueness}{{\circled{$\hspace{0.5px}!$}}}
\newcommand{\rightimp}{{\circled{$\Rightarrow$}}}
\newcommand{\leftimp}{{\circled{$\Leftarrow$}}}

\DeclareMathOperator{\sign}{sign}
\DeclareMathOperator{\Cl}{Cl}
\DeclareMathOperator{\proj}{pr}
\DeclareMathOperator{\Arg}{Arg}
\DeclareMathOperator{\supp}{supp}
\DeclareMathOperator{\diag}{diag}
\DeclareMathOperator{\tr}{tr}
\DeclareMathOperator{\rank}{rank}
\DeclareMathOperator{\Lat}{Lat}
\DeclareMathOperator{\Lin}{Lin}
\DeclareMathOperator{\Ln}{Ln}
\DeclareMathOperator{\Orbit}{Orbit}
\DeclareMathOperator{\St}{St}
\DeclareMathOperator{\Seq}{Seq}
\DeclareMathOperator{\PSet}{PSet}
\DeclareMathOperator{\MSet}{MSet}
\DeclareMathOperator{\Cyc}{Cyc}
\DeclareMathOperator{\Hom}{Hom}
\DeclareMathOperator{\End}{End}
\DeclareMathOperator{\Aut}{Aut}
\DeclareMathOperator{\Ker}{Ker}
\DeclareMathOperator{\Def}{def}
\DeclareMathOperator{\Alt}{Alt}
\DeclareMathOperator{\Sim}{Sim}
\DeclareMathOperator{\Int}{Int}
\DeclareMathOperator{\grad}{grad}
\DeclareMathOperator{\sech}{sech}
\DeclareMathOperator{\csch}{csch}
\DeclareMathOperator{\asin}{\sin^{-1}}
\DeclareMathOperator{\acos}{\cos^{-1}}
\DeclareMathOperator{\atan}{\tan^{-1}}
\DeclareMathOperator{\acot}{\cot^{-1}}
\DeclareMathOperator{\asec}{\sec^{-1}}
\DeclareMathOperator{\acsc}{\csc^{-1}}
\DeclareMathOperator{\asinh}{\sinh^{-1}}
\DeclareMathOperator{\acosh}{\cosh^{-1}}
\DeclareMathOperator{\atanh}{\tanh^{-1}}
\DeclareMathOperator{\acoth}{\coth^{-1}}
\DeclareMathOperator{\asech}{\sech^{-1}}
\DeclareMathOperator{\acsch}{\csch^{-1}}

\newcommand*{\scriptA}{{\mathcal{A}}}
\newcommand*{\scriptB}{{\mathcal{B}}}
\newcommand*{\scriptC}{{\mathcal{C}}}
\newcommand*{\scriptD}{{\mathcal{D}}}
\newcommand*{\scriptF}{{\mathcal{F}}}
\newcommand*{\scriptH}{{\mathcal{H}}}
\newcommand*{\scriptK}{{\mathcal{K}}}
\newcommand*{\scriptL}{{\mathcal{L}}}
\newcommand*{\scriptM}{{\mathcal{M}}}
\newcommand*{\scriptP}{{\mathcal{P}}}
\newcommand*{\scriptQ}{{\mathcal{Q}}}
\newcommand*{\scriptR}{{\mathcal{R}}}
\newcommand*{\scriptT}{{\mathcal{T}}}
\newcommand*{\scriptU}{{\mathcal{U}}}
\newcommand*{\scriptX}{{\mathcal{X}}}
\newcommand*{\Cnk}[2]{\left(\begin{matrix}#1\\#2\end{matrix}\right)}
\newcommand*{\im}{{\mathbf i}}
\newcommand*{\id}{{\mathrm{id}}}
\newcommand*{\compl}{^\complement}
\newcommand*{\dotprod}[2]{{\left\langle{#1},{#2}\right\rangle}}
\newcommand\matr[1]{\left(\begin{matrix}#1\end{matrix}\right)}
\newcommand\matrd[1]{\left|\begin{matrix}#1\end{matrix}\right|}
\newcommand\arr[2]{\left(\begin{array}{#1}#2\end{array}\right)}

\DeclareMathOperator{\divby}{\scalebox{1}[.65]{\vdots}}
\DeclareMathOperator{\toto}{\rightrightarrows}
\DeclareMathOperator{\ntoto}{\not\rightrightarrows}

\newcommand{\undercolorblack}[2]{{\color{#1}\underline{\color{black}#2}}}
\newcommand{\undercolor}[2]{{\colorlet{tmp}{.}\color{#1}\underline{\color{tmp}#2}}}

\usepackage{adjustbox}

\geometry{margin=1in}
\usepackage{fancyhdr}
\pagestyle{fancy}
\fancyfoot[L]{}
\fancyfoot[C]{Иванов Тимофей}
\fancyfoot[R]{\pagename\ \thepage}
\fancyhead[L]{}
\fancyhead[R]{\leftmark}
\renewcommand{\sectionmark}[1]{\markboth{#1}{}}

\setcounter{tocdepth}{5}
\usepackage{amsthm}
\usepackage{chngcntr}

\theoremstyle{definition}
\newtheorem{definition}{Определение}
\counterwithin*{definition}{section}

\theoremstyle{plain}
\newtheorem{theorem}{Теорема}
\counterwithin*{theorem}{section} % Without changing appearance
\newtheorem{lemma}{Лемма}
\counterwithin*{lemma}{section}
\newtheorem{corollary}{Следствие}[theorem]
\counterwithin{corollary}{theorem} % Changing appearance
\counterwithin{corollary}{lemma}
\newtheorem*{claim}{Утверждение}
\newtheorem{property}{Свойство}[definition]

\theoremstyle{remark}
\newtheorem*{remark}{Замечание}
\newtheorem*{example}{Пример}


%\renewcommand\qedsymbol{$\blacksquare$}

\counterwithin{equation}{section}


\fancyhead[L]{Математический анализ}

\let\eps\varepsilon

\undef\limsup
\DeclareMathOperator*{\limsup}{\overline{\lim}}
\undef\liminf
\DeclareMathOperator*{\liminf}{\underline{\lim}}

\let\tmp\varphi
\let\varphi\phi
\let\phi\tmp
\undef\tmp

\begin{document}
    \tableofcontents
    \begin{definition}
        Пусть $E\subset \mathbb R^n$ $f\colon E\to[0;+\infty]$. \textbf{Подграфиком} $f$ называется множество
        $$
        Q_f=\{(x;y)\in\mathbb R^{n+1}\mid x\in E,0\leqslant y\leqslant f(x)\}
        $$
    \end{definition}
    \begin{definition}
        Пусть $E\subset \mathbb R^n$ $f\colon E\to\overline{\mathbb R}$. \textbf{Графиком} $f$ называется множество
        $$
        \Gamma_f=\{(x;f(x))\in\mathbb R^{n+1}\mid x\in E\}
        $$
    \end{definition}
    \begin{remark}
        Отличается от общего определения тем, что $\Gamma_f\subset\mathbb R^{n+1}$
    \end{remark}
    \begin{theorem}[О мере графика]
        \label{О мере графика}
        Пусть $E\in\mathbb A_n$, $f\in S(E)$. Тогда $\Gamma_f\in\mathbb A_{n+1}$ и $\mu_{n+1}\Gamma_f=0$.
    \end{theorem}
    \begin{proof}
        \begin{itemize}
            \item Сначала разберём случай, когда $\mu E<+\infty$. Заключим $\Gamma_f$ в множество сколь угодно малой меры. Зафиксируем $\eps>0$. Пусть
            $$
            e_k=E(k\eps<f(k+1)\eps)
            $$
            Тогда
            $$
            E=E(|f|=+\infty)\cup\bigcup\limits_{k\in\mathbb Z}e_k
            $$
            Тогда
            $$
            \Gamma_f=\bigcup\limits_{k\in\mathbb Z}\Gamma_{f\big|_{e_k}}\subset\bigcup\limits_{k\in\mathbb Z}e_k\times[k\eps;(k+1)\eps)=H_\eps
            $$
            Заметим, что
            $$
            \mu_{n+1}H_\eps=\sum\limits_{k\in\mathbb Z}\mu_ne_k\cdot\eps\leqslant\mu_nE\eps
            $$
            По критерию измеримости утверждение теоремы верно.
            \item Теперь пусть $\mu E=+\infty$. По $\sigma$-конечности $\mu_n$
            $$
            E=\bigcup\limits_{j=1}^\infty E_j\qquad\mu_nE_j<+\infty
            $$
            А значит $f\Big|_{E_j}$ имеет измеримый график нулевой меры, а поскольку
            $$
            \Gamma_f=\bigcup\limits_{j=1}^\infty \Gamma_{f\big|_{E_j}}
            $$
            Верно требуемое.
        \end{itemize}
    \end{proof}
    \begin{theorem}
        \label{О мере подграфика}
        Пусть $E\in\mathbb A_n$, $f\colon E\to[0;+\infty]$. Тогда измеримость $f$ и её подграфика равносильны и в случае измеримости имеет место равенство
        $$
        \mu_{n+1}Q_f=\int_Ef~\mathrm d\mu_n
        $$
    \end{theorem}
    \begin{proof}
        Пусть нам известна измеримость подграфика. Тогда искомая формула следует из принципа Кавальери:
        $$
        Q_f(x)=\begin{cases}
            \varnothing & x\notin E\\
            [0;f(x)\rangle & x\in E
        \end{cases}
        $$
        Отсюда
        $$
        \mu_1Q_f(x)=\begin{cases}
            0 & x\notin E\\
            f(x) & x\in E
        \end{cases}
        $$
        Отсюда если $Q_f$ измеримо, то формула следует из принципа Кавальери. А также в принципе Кавальери в качестве следствия был факт, что функция $x\mapsto\mu_1Q_f(x)$ измерима, а значит и $f$ измерима как сужение $x\mapsto\mu_1Q_f(x)$ на $E$.\\
        Осталось доказать, что если $f$ измерима, то её подграфик измерим. Рассмотрим случаи:
        \begin{enumerate}
            \item $f$ простая.
            $$
            f=\sum\limits_{k=1}^Nc_k\chi_{A_k}\qquad A_k\in\mathbb A_n,c_k\in[0;+\infty)
            $$
            Можно считать, что $A_k$ дизъюнктны. И ещё можно считать, что $A_k\subset E$ и в объединении дают $E$. Тогда
            $$
            Q_f=\bigsqcup_{k=1}^NA_k\times[0;c_k]
            $$
            Отсюда следует измеримость.
            \item Общий случай: $f$ произвольная неотрицательная измеримая функция. Приблизим её возрастающей последовательность простых функций $\phi_n$. Проверим, что 
            $$
            Q_f=\bigcup Q_{\phi_n}\cup\Gamma_f
            $$
            Тогда мы докажем искомое.
            \begin{itemize}
                \item[$\supset$] ясно т.к. $\phi_n\leqslant f\Rightarrow Q_{\phi_n}\subset Q_f$.
                \item[$\subset$] рассмотрим $(x;y)\in Q_f$. То есть $x\in E$, $y\in[0;f(x)]$. Если $y=f(x)$, то понятно. Иначе
                $$
                \exists N\in\mathbb N~y<\phi_N(x)\Rightarrow \exists N~(x;y)\in Q_{\phi_N}
                $$
            \end{itemize}
        \end{enumerate}
    \end{proof}
    \begin{remark}
        Условие измеримости $E$ существенно. Если $f\equiv0$ не неизмеримом множестве, например, то $Q_f\in\mathbb A_{n+1}$ и $\mu_{n+1}Q_f=0$.
    \end{remark}
    \begin{theorem}[Теорема Тонелли]
        \label{Теорема Тонелли}
        Пусть $E\subset\mathbb R^{n+m}$, $f\in S(E\to[0;+\infty])$. Тогда
        \begin{enumerate}
            \item При почти всех $x\in\mathbb R^n$ функция $f(x;\bullet)\in S(E(x))$.
            \item Пусть $I(x)=\int_{E(x)}f(x;y)~\mathrm dy$. Тогда $I(x)\in S(\mathbb R^n)$.
            \item
            $$\int_E f~\mathrm d\mu_{n+m}=\int_{\mathbb R^n}I(x)~\mathrm dx$$
        \end{enumerate}
    \end{theorem}
    \begin{proof}
        По теореме \ref{О мере подграфика}, что $Q_f\in\mathbb A_{n+m+1}$ и
        $$
        \mu_{n+m+1}Q_f=\int_E f~\mathrm d\mu_{n+m}
        $$
        Воспользуемся принципом Кавальери:
        $$
        \mu_{n+m+1}Q_f=\int_{\mathbb R^n}\mu_{m+1}Q_f(x)~\mathrm dx
        $$
        Заметим, что
        \[\begin{split}
            Q_f(x)&=\left\{(y;z)\in\mathbb R^{m+1}\mid (x;y;z)\in Q_f\right\}=\\
            &=\left\{(y;z)\in\mathbb R^{m+1}\mid (x;y)\in E,z\in[0;f(x;y)]\right\}=\\
            &=\left\{(y;z)\in\mathbb R^{m+1}\mid y\in E(x),z\in[0;f(x;y)]\right\}
        \end{split}\]
        Да это же подграфик $f(x;\bullet)$!
        \begin{enumerate}
            \item По теореме Кавальери при почти всех $x\in\mathbb R^n$ $Q_f(x)$ измеримо, а значит мы доказали первое утверждение по теореме \ref{О мере подграфика}.
            \item
            $$
            \mu_{m+1}Q_f(x)=\mu_{m+1}Q_{f(x;\bullet)}\overset{\ref{О мере подграфика}}=\int_{E(x)}f(x;y)~\mathrm dy=I(x)
            $$
            Отсюда $I(x)$ измерима по всё тому же принципу Кавальери.
            \item Приравняем два выражения для $\mu_{n+m+1}Q_f$.
        \end{enumerate}
    \end{proof}
    \begin{theorem}[Теорема Фубини]
        \label{Теорема Фубини}
        \label{Выражение кратного интеграла через повторный}
        Пусть $E\subset\mathbb R^{n+m}$, $f\in L(E)$. Тогда
        \begin{enumerate}
            \item При почти всех $x\in\mathbb R^n$ функция $f(x;\bullet)\in L(E(x))$.
            \item Пусть $I(x)=\int_{E(x)}f(x;y)~\mathrm dy$. Тогда $I(x)\in L(\mathbb R^n)$.
            \item
            $$\int_E f~\mathrm d\mu_{n+m}=\int_{\mathbb R^n}I(x)~\mathrm dx$$
        \end{enumerate}
    \end{theorem}
    \begin{proof}
        Применим \hyperref[Теорема Тонелли]{теорему Тонелли} для $f_+$ и $f_-$. Пусть $I^\pm=\int_{E(x)}f_\pm(x;y)~\mathrm dy$. По теореме Тонелли
        $$
        \int_E f_\pm~\mathrm d\mu_{n+m}=\int_{\mathbb R^n}I^\pm(x)~\mathrm dx<+\infty
        $$
        Учитывая $f_{\pm}(x;\bullet)=(f(x;\bullet))_\pm$, имеем
        $$
        I^+-I^-=I\in L(\mathbb R^n)
        $$
        При почти всех $x$ $I^\pm(x)<+\infty$, а значит при почти всех $x$ $f_\pm\in L(E(x))$ отсюда $f\in L(E(x))$.
    \end{proof}
    \begin{remark}
        В \hyperref[Теорема Тонелли]{теореме Тонелли} все условия можно ослабить:
        \begin{enumerate}
            \item Если $f\in S(E)$ (не важен знак), то при почти всех $x\in \mathbb R^n$ $f(x;\bullet)\in S(E(x))$.
            \item Если $I(x)$ существует почти во всех $x\in\mathbb R^n$, то $I\in S(\mathbb R^n)$
            \item Если существует $\int_Ef~\mathrm d\mu_{n+m}\in\overline{\mathbb R}$, то верно условие пункта 2 и
            $$
            \int_Ef~\mathrm d\mu_{n+m}\in\overline{\mathbb R}=\int_{\mathbb R^n}I(x)~\mathbb dx
            $$
        \end{enumerate}
        Доказывается всё это как в теореме Фубини.
    \end{remark}
    \begin{remark}
        $$
        \left\{\begin{aligned}
            &\forall x\in\mathbb R^n~f(x;\bullet)\in S(E(x))\\
            &\forall y\in\mathbb R^m~f(\bullet;m)\in S(E(y))
        \end{aligned}\right.\not\Rightarrow f\in S(E)
        $$
        Серпинский построил пример такого неизмеримого $E\subset\mathbb R^2$, что $E$ пересекается с любой прямой не более чем в двух точках. Мы говорить о нём не будем, т.к. он довольно сложен.
    \end{remark}
    \begin{definition}
        Пусть $X\subset\mathbb R^n$, $Y\subset\mathbb R^m$, $f\colon X\to\mathbb R$, $g\colon Y\to\mathbb R$. Тогда
        $$
        f\otimes g\colon\substack{X\times Y\to\mathbb R\\(x;y)\mapsto f(x)g(y)}
        $$
    \end{definition}
    \begin{lemma}
        Если $f\in S(X)$, $g\in S(Y)$, то $f\otimes g\in S(X\times Y)$.
    \end{lemma}
    \begin{proof}
        Пусть
        $$
        \tilde f(x;y)=f(x)\qquad\tilde g(x;y)=g(y)
        $$
        Докажем, что $\tilde f$ и $\tilde g$ измеримы, тогда $f\otimes g$ будет измеримо как произведение измеримых.\\
        $$(X\times Y)(\tilde f>a)=X(\tilde f>0)\times Y$$
        Левое измеримо по измеримости $f$, правое~--- потому что, а произведение измеримы измеримо.
    \end{proof}
    \begin{corollary}
        Пусть $X\subset\mathbb R^n$, $Y\subset\mathbb R^m$. Если
        $$
        \left[\begin{aligned}
            &f\in S(X\to[0;+\infty])\land g\in S(Y\to[0;+\infty])\\
            &f\in L(X)\land g\in L(Y)
        \end{aligned}\right.
        $$
        То
        $$
        \int_{X\times Y}f\otimes g~\mathrm d\mu_{n+m}=\int_Xf~\mathrm d\mu_n\int_Yg~\mathrm d\mu_m
        $$
    \end{corollary}
    \begin{proof}
        В первом случае нет сомнений в существовании интегралов. Пусть $E=X\times Y$. Тогда
        $$
        \int_Ef\otimes g~\mathrm d\mu_{n+m}=\int_X\left(\int_Yf(x)g(y)~\mathrm dy\right)~\mathrm dx
        $$
        так как $E(x)=\begin{cases}
            \varnothing&x\notin X\\
            Y&x\in X
        \end{cases}$.\\
        А почему то же самое верны для произвольного знака, если интегралы от них конечны? Ну, чтобы применить теорему Фубини, надо проверить суммируемость $f\otimes g$. Ну, смотрите. По доказанному
        $$
        \int_{X\times Y}|f\otimes g|~\mathrm d\mu_{n+m}=\int_X|f|~\mathrm d\mu_n\int_Y|g|~\mathrm d\mu_m
        $$
        По условию оба этих интеграла конечны, значит $|f\otimes g|$ суммируема, а суммируемость функции равносильна суммируемости её модуля.
    \end{proof}
    \begin{remark}
        Мы знаем, что в условиях теорем \hyperref[Теорема Тонелли]{Тонелли} и \hyperref[Теорема Фубини]{Фубини} верно
        $$
        \int_E f~\mathrm d\mu_{n+m}=\int_{\mathbb R^n}\left(\int_{E(x)}f(x;y)~\mathrm dy\right)~\mathrm dx
        $$
        Тривиально, то же можно записать, поменяв $x$ и $y$ ролями.
        $$
        \int_E f~\mathrm d\mu_{n+m}=\int_{\mathbb R^n}\left(\int_{E(y)}f(x;y)~\mathrm dx\right)~\mathrm dy
        $$
        А значит два повторных интеграла равны.
    \end{remark}
    \begin{example}
        Повторные интегралы могут быть не равны
        $$
        f(x;y)=\frac{x^2-y^2}{(x^2+y^2)^2}\qquad E=[-1;1]^2
        $$
        Тогда
        $$
        \int\limits_{-1}^1 \frac{x^2-y^2}{(x^2+y^2)^2}~\mathrm dy=\frac y{x^2+y^2}\bigg|_{y=-1}^1=\frac2{x^2+1}
        $$
        $$
        \int\limits_{-1}^1 \int\limits_{-1}^1 \frac{x^2-y^2}{(x^2+y^2)^2}~\mathrm dy~\mathrm dx=\int\limits_{-1}^1 \frac2{x^2+1}~\mathrm dx=4\atan x\bigg|_0^1=\pi
        $$
        Другой повторный интеграл будет равен $-\pi$, как несложно заметить.
    \end{example}
    \begin{example}
        Неверно, что если повторные интегралы равны, то двойной существует.
        $$
        g(x;y)=\frac{2xy}{(x^2+y^2)^2}\qquad E=[-1;1]^2
        $$
        Поскольку функция $g$ нечётна по каждой переменной, оба повторных интеграла равны нулю. Отсюда если двойной интеграл существует, то равен нулю.\\
        Докажем, что он не существует. Для этого докажем, что $g$ не суммируема. Попробуем проинтегрировать $|g|$:
        $$
        \int\limits_{-1}^1\int\limits_{-1}^1\frac{|2xy|}{(x^2+y^2)^2}~\mathrm dy~\mathrm dx=
        4\int\limits_0^1\int\limits_0^1\frac{|2xy|}{(x^2+y^2)^2}~\mathrm dy~\mathrm dx=
        4\int\limits_0^1\frac{-x}{x^2+y^2}\bigg|_{y=0}^1~\mathrm dx=
        4\int\limits_0^1\frac{-x}{x^2+1}+\frac1x~\mathrm dx=+\infty
        $$
        Отсюда $g$ не суммируема. А значит нулю её интеграл не равен, то есть он не существует.
    \end{example}
    \begin{definition}
        Пусть $E\subset\mathbb R^{n+m}$ \textbf{Проекцией} $E$ на первое координатное пространство называется
        $$
        \operatorname{Pr}_1E=\left\{x\in\mathbb R^n\mid E(x)\neq\varnothing\right\}
        $$
    \end{definition}
    \begin{remark}
        Проекция измеримого множества может быть быть неизмеримой (достаточно добавить к измеримому двумерному множеству неизмеримое одномерное).
    \end{remark}
    \begin{definition}
        Множество
        $$
        \operatorname{Pr}^*_1E=\left\{x\in\mathbb R^n\mid \mu_m E(x)>0\right\}
        $$
        называется \textbf{существенной проекцией} множества $E$.
    \end{definition}
    \begin{property}
        Существенная проекция измерима. (Как Лебегово множество функции $\mu_m E(\bullet)$).
    \end{property}
    \begin{property}
        При $f$ подходящем под теоремы \hyperref[Теорема Тонелли]{Тонелли} и \hyperref[Теорема Фубини]{Фубини} верно
        $$
        \int_E f~\mathrm d\mu_{n+m}=\int_{\Pr^*_1}I(x)~\mathrm dx
        $$
    \end{property}
    \begin{remark}
        Теоремы \hyperref[Теорема Тонелли]{Тонелли} и \hyperref[Теорема Фубини]{Фубини} можно применять несколько раз.
    \end{remark}
    \begin{definition}
        Пусть $(X;\mathbb A;\mu)$ и $(Y;\mathbb B;\nu)$~--- пространства с мерами. Пусть
        $$
        \mathbb A\odot\mathbb B=\{A\times B\mid A\in\mathbb A,B\in\mathbb B\}
        $$
        Тогда $\mathbb A\odot\mathbb B$ является полукольцом, а
        $$
        \pi_0\colon A\times B\to\mu A\nu B
        $$
        является мерой на $\mathbb A\oplus\mathbb B$. Тогда $\pi$~--- стандартное распространение $\pi_0$ на $\sigma$-алгебру $\mathbb C$ называется произведением мер $\mu$ и $\nu$.\\
        Обозначения:
        $$
        \mathbb C=\mathbb A\otimes\mathbb B\qquad \pi=\mu\times\nu
        $$
    \end{definition}
    \begin{remark}
        Доказывать корректность определения мы не будем.
    \end{remark}
    \begin{property}
        $$
        \mu_{n+m}=\mu_n\times\mu_m
        $$
    \end{property}
    \begin{property}
        Если $\mu$ и $\nu$ являются $\sigma$-конечными, то $\mu\times\nu$~--- тоже.
    \end{property}
    \begin{property}
        Произведение мер ассоциативно.
    \end{property}
    \begin{property}
        Все теоремы этого параграфа с доказательствами верны для полных $\sigma$-конечных мер.
    \end{property}
    \begin{theorem}[Теорема Тонелли для абстрактных пространств с мерой]
        Пусть $E\subset X\times Y$, $f\in S_{\mathbb A\otimes\mathbb B}(E\to[0;+\infty])$. Тогда
        \begin{enumerate}
            \item При почти всех $x\in x$ функция $f(x;\bullet)\in S_{\mathbb B}(E(x))$.
            \item Пусть $I(x)=\int_{E(x)}f(x;\bullet)~\mathrm d\nu$. Тогда $I(x)\in S_{\mathbb A}(X)$.
            \item
            $$\int_E f~\mathrm d(\mu\times\nu)=\int_{X}I(x)~\mathrm d\mu$$
        \end{enumerate}
    \end{theorem}
    \paragraph{Замена переменной в интеграле.}
    \begin{theorem}[Общая схема замены переменной в интеграле]
        \label{Общая схема замены переменной в интеграле}
        Пусть $(X;\mathbb A;\mu)$, $(Y;\mathbb B;\nu)$~--- пространства с мерами. Пусть $h\in S_{\mathbb A}(X\to[0;+\infty])$, $\Phi\colon X\to Y$ и
        $$
        \forall B\in\mathbb B~\Phi^{-1}(B)\in\mathbb A\qquad \forall B\in\mathbb B~\nu B=\int_{\Phi^{-1}(B)}h~\mathrm d\mu
        $$
        Пусть $f\in S_{\mathbb B}(Y)$. Тогда
        \begin{enumerate}
            \item $f\circ\Phi\in S_{\mathbb A}(X)$.
            \item
            $$
            \int_Y f~\mathrm d\nu=\int_X(f\circ\Phi)h~\mathrm d\mu
            $$
            (Трактовка стандартная: интегралы существуют или нет одновременно, если существуют, то равны.)
        \end{enumerate}
    \end{theorem}
    \begin{proof}
        \begin{enumerate}
            \item Рассмотрим Лебегово множество
            $$
            X(f\circ\Phi>a)=\{x\in X\mid f(\Phi(x))>a\}=\{x\in X\mid \Phi(x)\in Y(f>a)\}=\Phi^{-1}(Y(f>a))
            $$
            По условию $Y(f>a)\in \mathbb B$, а значит $\Phi^{-1}\in\mathbb A$.
            \item Разберём случаи
            \begin{enumerate}[a.]
                \item $f=\chi_B\mid B\in\mathbb B$. Тогда
                $$
                (\chi_B\circ\Phi)(x)=\begin{cases}
                    1 & x\in\Phi^{-1}(B)\\0 & x\notin\Phi^{-1}(B)
                \end{cases}=\chi_{\Phi^{-1}(B)}(x)
                $$
                Тогда
                $$
                \int_{Y}\chi_B~\mathrm d\nu=\nu B=\int_{\Phi^{-1}(B)}h~\mathrm d\nu=\int_{X}\underbrace{\chi_{\Phi^{-1}(B)}}_{f\circ\Phi}h~\mathrm d\nu
                $$
                \item По линейности равенство верно для простых функций.
                \item Для положительных измеримых функций рассмотрим последовательность $\phi_n$, возрастающую к $f$ и перейдём к пределу в равенстве
                $$
                \int_Y\phi_n~\mathrm d\nu=\int_X(\phi_n\circ\Phi)h~\mathrm d\mu
                $$
                по теореме Леви.
                \item Для произвольных измеримых функций рассмотрим $f_{\pm}$
                $$
                \int_Yf_\pm~\mathrm d\nu=\int_X(\phi_n\circ\Phi)_\pm h~\mathrm d\mu=\int_X(\phi_n\circ\Phi h)_\pm~\mathrm d\mu
                $$
            \end{enumerate}
        \end{enumerate}
    \end{proof}
    \begin{remark}
        В условиях теоремы \ref{Общая схема замены переменной в интеграле} суммируемость $f$ по $\nu$ равносильная суммируемости $(f\circ\Phi)h$ по $\mu$
    \end{remark}
    \begin{corollary}
        В условиях теоремы \ref{Общая схема замены переменной в интеграле} если $B\in\mathbb B$, $f\in S_{\mathbb B}(B)$, то
        $$
        \int_B f~\mathrm d\nu=\int_{\Phi^{-1}(B)}(f\circ\Phi)h~\mathrm d\mu
        $$
    \end{corollary}
    \begin{proof}
        Продолжим $f$ нулём на $Y\setminus B$.
    \end{proof}
    \begin{definition}
        В условии теоремы \ref{Общая схема замены переменной в интеграле} $\nu$ называется $h$-взвешенным $\Phi$-\textbf{образом меры} $\mu$.
    \end{definition}
    \begin{remark}
        Пусть
        $$
        \mathbb A^*=\{\Phi^{-1}(B)\mid B\in\mathbb B\}
        $$
        Нетрудно заметить, что это $\sigma$-алгебра.\\
        В условиях теоремы \ref{Общая схема замены переменной в интеграле} $\mathbb A^*\subset\mathbb A$.\\
        Пусть
        $$
        \mathbb B^*=\{B\subset Y\mid \Phi^{-1}\in\mathbb A\}
        $$
        тогда условиях теоремы \ref{Общая схема замены переменной в интеграле} $\mathbb B\subset\mathbb B^*$.
    \end{remark}
    \begin{claim}
        Если
        $$
        \nu B=\int_{\Phi^{-1}(B)}h~\mathrm d\mu
        $$
        То $\nu$~--- мера на $\mathbb B$.
    \end{claim}
    \begin{proof}
        Остаётся как несложное упражнение читателю.
    \end{proof}
    \begin{example}
        $h\equiv1$~--- невзвешенный образ меры.
        $$
        \nu B=\mu\Phi^{-1}(B)\Rightarrow \int_Yf~\mathrm d\nu=\int_Xf\circ\Phi~\mathrm d\mu
        $$
    \end{example}
    \begin{example}
        $X=Y$, $\mathbb A=\mathbb B$, $\Phi=\mathrm{id}$.
        $$
        \nu A=\int_A h~\mathrm d\mu\Rightarrow \int_X f~\mathrm d\nu=\int_Xfh~\mathrm d\mu
        $$
        Тогда пишут $\mathrm d\nu=h\mathrm d\mu$.
    \end{example}
    \begin{definition}
        Если
        $$\nu A=\int_A h~\mathrm d\mu$$
        то $h$ называется \textbf{плотностью} меры $\nu$ относительны меры $\mu$.
    \end{definition}
    \begin{property}
        Если $h=\tilde h$ $\mu$-почти везде, то $\nu=\tilde\nu$. Для $\sigma$-конечных мер верно и обратное.\\
        Без доказательства.
    \end{property}
    \begin{theorem}[Критерий плотности]
        \label{Критерий плотности}
        Пусть даны $X,\mathbb A$ и $\mu$ и $\nu$~--- меры на $\mathbb A$, $h\colon S(X\to[0;+\infty])$. Тогда следующие утверждения равносильны:
        \begin{enumerate}
            \item $h$~--- плотность $\nu$ относительны $\mu$.
            \item
            $$
            \forall A\in\mathbb A~\mu A\inf\limits_Ah\leqslant\nu A\leqslant \mu A\sup\limits_Ah
            $$
        \end{enumerate}
    \end{theorem}
    \begin{proof}
        \begin{itemize}
            \item Из первого во второе ясно из оценки интеграла.
            \item Рассмотрим
            $$
            A=A(h=0)\cup A(0<h<+\infty)\cup A(h=+\infty)
            $$
            Равенство есть для первой части:
            $$\nu A(h=0)=\int_{A(h=0)} h~\mathrm d\mu$$
            так как левое равно нулю по условию второго утверждения, а правое~--- потому что функция тождественный ноль.\\
            также очевидно равенство есть для третьей части:
            $$\nu A(h=+\infty)=\begin{cases}
                +\infty & \mu A>0\\0 & \mu A\equiv0
            \end{cases}=\int_{A(h=+\infty)} h~\mathrm d\mu$$
            Далее можно считать $0<h<+\infty$ на $A$.\\
            Рассмотрим $q\in(0;1)$. Пусть
            $$
            A_j=A(q^j\leqslant h<q^{j-1})
            $$
            Очевидно, $A_j\in\mathbb A$ и $\bigsqcup\limits_{j\in\mathbb Z}A_j=A$. Нам известно, что
            $$
            q^j\mu A_j\leqslant\nu A_j\leqslant q^{j-1}\mu A_j
            $$
            А ещё из оценки интеграла
            $$
            q^j\mu A_j\leqslant\int_{A_j}h~\mathrm d\mu\leqslant q^{j-1}\mu A_j
            $$
            Отсюда
            \[\begin{split}
                q\int_Ah~\mathrm d\mu&=q\sum\limits_j\int_{A_j}h~\mathrm d\mu\leqslant\\
                &\leqslant\sum\limits_jq^j\mu A_j\leqslant\sum\limits_j\nu A_j=\\
                &=\fbox{\nu A}\leqslant\sum\limits_j q^{j-1}\mu A_j\leqslant\\
                &\leqslant\frac1q\sum\limits_j\int_{A_j}h~\mathrm d\mu=\\
                &=\frac1q\int_Ah~\mathrm d\mu
            \end{split}\]
            Если взять начало, конец и то, что в квадратике, после чего устремить $q$ к единицу, то получим искомое.
        \end{itemize}
    \end{proof}
    \subparagraph{Интеграл по дискретной мере.}
    \begin{example}
        $\delta$-мера.\\
        Пусть $X$~--- пространство, $E\subset X$, $a\in X$, тогда, напомним, $\delta$-мера~--- это
        $$\delta_a(E)=\begin{cases}
            1 & a\in E\\
            0 & a\notin E
        \end{cases}$$
        В качестве сигма-алгебры выступает $2^X$. Пусть $f\colon E\to\overline{\mathbb R}$. Что тогда такое интеграл по этой мере?
        $$
        \int_Ef~\mathrm d\delta_a=\begin{cases}
            0 & a\notin E\\
            \int_{\{a\}}f~\mathrm d\delta_a=f(a) & a\in E
        \end{cases}
        $$
    \end{example}
    \begin{lemma}
        Пусть $\mu$~--- считающая мера на $X$, $E\subset X$, $f\colon E\to\overline{\mathbb R}$. Тогда
        $$
        \int_Ef~\mathrm d\mu=\sum\limits_Ef
        $$
        Интеграл и сумма существуют или не существуют одновременно, если существуют, то равны.
    \end{lemma}
    \begin{proof}
        будем постепенно усложнять $f$.
        \begin{enumerate}
            \item Сначала докажем для характеристической функции $f=\chi_A$, $A\subset E$. Тогда
            $$
            \int_E\chi_A~\mathrm d\mu=\mu A=\sum\limits_A1=\sum\limits_A\chi_A
            $$
            \item По линейности равенство верно для простых функций $f$.
            \item $f\geqslant0$. Разберём два случая:
            \begin{itemize}
                \item Пусть $\int_Ef~\mathrm d\mu<+\infty$. Тогда
                $$
                \int_Ef~\mathrm d\mu=\sup\limits_{\substack{A\subset E\\\mu A<+\infty}}\int_Af~\mathrm d\mu
                $$
                Условие $\mu A<+\infty$ значит что $|A|<+\infty$ (у нас считающая мера). А на конечном множестве положительная функция простая:
                $$
                \int_Ef~\mathrm d\mu=\sup\limits_{\substack{A\subset E\\\mu A<+\infty}}\sum\limits_Af
                $$
                Справа написано буквально определение $\sum\limits_Ef$ (там супремум частичных сумм).
                \item Пусть $\int_Ef~\mathrm d\mu=+\infty$. По определению интеграла неотрицательной функции
                $$
                \forall N>0~\exists\text{простая }\phi\leqslant f\text{ на }E~\int_E\phi~\mathrm d\mu\geqslant N
                $$
                При этом
                $$
                \sum\limits_Ef\geqslant\sum\limits_E\phi=\int_E\phi~\mathrm d\mu\geqslant N
                $$
            \end{itemize}
            \item Дальше надо рассмотреть $f_+$ и $f_-$, там всё понятно. Если в одной части нет одновременно двух бесконечностей, то в другой~--- тоже.
        \end{enumerate}
    \end{proof}
    \begin{remark}
        Причём тут замена переменной?
    \end{remark}
    \begin{corollary}
        Пусть $h\colon X\to[0;+\infty]$, $\mu$~--- считающая на $X$, $\nu$~--- дискретная мера с весовой функцией $h$:
        $$
        \forall B\subset X~\nu B=\sum\limits_Bh
        $$
        Пусть $f\colon E\to\overline{\mathbb R}$. Тогда
        $$
        \int_Ef~\mathrm d\nu=\sum\limits_Ef\cdot h
        $$
    \end{corollary}
    \begin{proof}
        По только что доказанной лемме
        $$
        \nu B=\int_Bh~\mathrm d\mu
        $$
        А тогда $h$~--- плотность $\nu$ относительно $\mu$. Тогда по теореме \ref{Общая схема замены переменной в интеграле}
        $$
        \int_Ef~\mathrm d\nu\overset{\ref{Общая схема замены переменной в интеграле}}=\int_Efh~\mathrm d\mu=\sum\limits_Efh
        $$
    \end{proof}
    \begin{example}
        Если $X=\mathbb N$, $\mu$~--- считающая мера, т
        $$
        \int_{\mathbb N}f~\mathrm d\mu=\sum\limits_{k\in\mathbb N}f(k)
        $$
        То есть суммируемость $f$ равносильна абсолютной сходимости ряда $f(k)$.
    \end{example}
    \begin{example}
        Пусть $\{a_k\}_k$~--- не более чем счётный набор различных точек $X$, $\{h_k\}_k\subset[0;+\infty]$, $\nu B=\sum\limits_{k:a_k\in B}h_k$. Тогда
        $$
        \int_Ef~\mathrm d\nu=\sum\limits_{k:a_k\in B}f(a_k)h_k
        $$
        И опять суммируемость функции равносильна суммируемости семейства (на сей раз семейства $|f(a_k)|h_k$).\\
        Замечание: тут $h$ из следствия~--- это не совсем $h_k$. Тут $h_k=h(a_k)$
    \end{example}
    \subparagraph{Замена переменной в интеграле по мере Лебега}.
    \begin{claim}
        Пусть $G\in\mathbb A_n$. Тогда
        $$\mathbb A_n(G)=\{E\in\mathbb A_n\mid E\subset G\}$$
        является $\sigma$-алгеброй на $G$.
    \end{claim}
    \begin{proof}
        Очевидно.
    \end{proof}
    \begin{claim}
        $(G;\mathbb A_n(G);\mu\big|_{\mathbb A_n(G)})$~--- пространство с мерой.
    \end{claim}
    \begin{proof}
        Очевидно.
    \end{proof}
    \begin{remark}
        До конца параграфа $\mu$~--- мера Лебега.
    \end{remark}
    \begin{remark}
        Напоминание:\\
        Пусть $G,V$ открыты в $\mathbb R^n$. Тогда отображение $\Phi\colon G\to V$ называется \textbf{диффеоморфизмом}, если $\Phi$ гладкая биекция, обратная функция к которой тоже гладкая.\\
        При этом обычно $V$ опускается, и под <<диффеоморфизмом $\Phi$ на $G$ называется диффеоморфизм $G\to\Phi(G)$>>.\\
        Также заметим некоторые свойства: якобиан $\Phi$ нигде не равен нулю. При этом если $G$ открыто, $\Phi\colon G\to\mathbb R^n$ гладко и обратимо и $\det\Phi'(x)$ нигде не равен нулю, то $\Phi(G)$ открыто и $\Phi^{-1}\in C^{(1)}(\Phi(G))$.
    \end{remark}
    \begin{remark}
        Также мы знаем, что гладкая замена переводит измеримые множества в измеримые. Вопрос: чему равно $\mu\Phi(A)$? Мы знаем ответ для линейного и аффинного отображения (мера умножается на модуль определителя). При этом для линейного и аффинного отображения $\Phi'=\Phi$, а значит
        $$
        \mu\Phi(E)=|\det\Phi'|\mu E
        $$
        А что в более общем случае? Ну, запишем определение дифференцируемости $\Phi$:
        $$
        \Phi(x)=\underbrace{\Phi(x^0)+\Phi'(x)(x-x^0)}_{\widetilde\Phi_{x^0}(x)}+o(x-x^0)
        $$
        Если $\Phi=\widetilde\Phi$, то ответ мы знаем. При этом $\widetilde \Phi$ тем ближе к $\Phi$, чем ближе $x$ к $x^0$. Отсюда возникает предположение, что $|\det\Phi'(x^0)|$~--- плотность $\mu\Phi(\bullet)$ относительно $\mu$. И нам удастся это доказать для диффеоморфизма.
    \end{remark}
    \begin{theorem}[Преобразование меры Лебега при диффеоморфизме]
        \label{Преобразование меры Лебега при диффеоморфизме}
        Пусть $G\subset\mathbb R^n$ открыто, $\Phi$~--- диффеоморфизм на $G$. Тогда
        $$
        \forall E\in\mathbb A_n(G)~\mu\Phi(E)=\int_E|\det\Phi'|~\mathrm d\mu
        $$
    \end{theorem}
    \begin{proof}
        Why are we still here? Just to suffer.\\
        Для начала пусть $\nu(E)=\mu\Phi(E)$. Несложно проверить, что $\nu$~--- это мера на $\mathbb A_n(G)$.\\
        Теперь нам надо доказать, что $|\det\Phi'|$~--- плотность $\nu$  относительно $\mu$. Тогда по определению плотности мы победим. У нас был \hyperref[Критерий плотности]{критерий плотности}, который мы хотим применить. Что нам надо проверить для этого?
        $$
        \forall E\in\mathbb A_n(G)~\mu E\inf\limits_E|\det\Phi'|\leqslant\mu\Phi(E)\leqslant\mu E\sup\limits_E|\det\Phi'|
        $$
        Здесь есть два неравенства. Мы не хотим доказывать оба. Мы хотим сказать, что если правое доказать для \textbf{любого} диффеоморфизма $\Phi$, то из него будет следовать левое. Почему? Ну, применим правое к отображению $\Phi^{-1}$ и множеству $\Phi(E)$:
        $$
        \mu\Phi^{-1}(\Phi(E))\leqslant\mu\Phi(E)\sup\limits_{y\in\Phi(E)}|\det{\Phi^{-1}}'(y)|
        $$
        Левая штука~--- $\mu E$. С $\mu\Phi(E)$ делать нечего, а вот с тем, что после него~--- есть что.
        $$
        \sup\limits_{y\in\Phi(E)}|\det{\Phi^{-1}}'(y)|=\sup\limits_{y\in\Phi(E)}\frac1{|\det \Phi'(\Phi^{-1}(y))|}=\sup\limits_{x\in E}\frac1{|\det\Phi'(x)|}=\frac1{\inf\limits_{x\in E}|\det\Phi'(x)|}
        $$
        Кажется, это то, что мы хотели.\\
        Теперь наконец начнём доказывать правое неравенство, постепенно усложняя $E$.
        \begin{enumerate}
            \item Пусть $E=\Delta$~--- кубическая ячейка, $\overline\Delta\subset G$. Докажем неравенство от противного. Пусть
            $$
            \mu\Phi(\Delta)>\mu\Delta\sup\limits_\Delta|\det\Phi'|
            $$
            отсюда
            $$\exists C>\mu\Delta\sup\limits_\Delta|\det\Phi'|~\mu\Phi(\Delta)>C\mu\Delta$$
            Будем действовать методом половинного деления. Каждое ребро ячейки попилим пополам, получим $2^n$ ячеек. Хотя бы для одной из этих ячеек (обозначим её за $\Delta_1$) будет верно $\mu\Phi(\Delta_1)>C\mu\Delta_1$. Иначе можно было бы сложить эти неравенства, воспользоваться аддитивностью меры и прийти к противоречию. Сделаем так ещё неограниченное количество раз.\\
            Получим последовательность вложенных ячеек $\Delta_k\supset\Delta_{k+1}$, для каждой выполнено неравенство $\mu\Phi(\Delta_k)>C\mu\Delta_k$, при этом $\operatorname{diam}\Delta_k\to0$. Ну тогда, $\overline{\Delta_k}$ все имеют общую точку $x^0\in\bigcap\limits_{k=1}^\infty\overline{\Delta_k}\subset\overline{\Delta}$.\\
            Теперь будем усложнять $\Phi$))
            \begin{enumerate}
                \item Рассмотрим случай $\Phi'(x^0)=I$. Тогда
                $$
                \Phi(x)=\Phi(x^0)+(x-x^0)+o(x-x^0)
                $$
                С точностью до двух сдвигов, $\Phi$ почти тождественный оператор:
                $$
                \Theta(x)=\Phi(x)-\Phi(x^0)+x^0=x+o(x-x^0)
                $$
                Зафиксируем $\eps>0$ и подберём такое $\delta$ из определения $o(x-x^0)$, что
                $$
                \forall x\in B(x^0;\delta)~|\Theta(x)-x|\leqslant\frac{\eps}{\sqrt n}|x-x^0|
                $$
                Очень хорошо. Заметим, что в некотором номере $N$ $\overline{\Delta_{N}}\subset B(x^0;\delta)$. Пусть $\overline{\Delta_{N}}=[a;a+r\mathbb1]\ni x^0$. Тогда
                $$
                x\in\overline{\Delta_n}\Rightarrow|x-x^0|\leqslant r\sqrt n\Rightarrow |\Theta(x)-x|\leqslant\eps r
                $$
                Тогда
                $$
                \forall j\in[1:n]~|\Theta_j(x)-x_j|\leqslant\eps r
                $$
                А это значит, что
                $$a-\eps r\leqslant x_j-\eps r\leqslant\Theta_j(x)\leqslant x_j+\eps r\leqslant a+(1+\eps)r$$
                другими словами $\Theta(x)\in[a-\eps r\mathbb1;a+(1+\eps)r\mathbb1]$. обозначим этот куб буквой $\Pi$. Тогда $\Theta(\Delta_N)\subset\Pi$. Тогда
                $$
                \mu\Phi(\Delta_N)=\mu\Theta(\Delta_N)\leqslant\mu\Pi=(1+2\eps)^nr^n=(1+2\eps)^n\mu\Delta_N
                $$
                И это уже почти противоречие. Но мы его не хотим, давайте сначала возьмём произвольное $\Phi$, и там уже докопаемся до противоречия.
                \item Итак, пусть $\Phi$ произвольное. Пусть $S=(\Phi'(x^0))^{-1}$ ({\color{red} это линейный оператор}, его производная в любой точке~--- он сам). Пусть $\Psi=S\circ\Phi$. Ну, хорошо
                $$
                \Psi'(x^0)=\underbrace{S'(\Phi(x^0))}_S\Phi'(x^0)=S\Phi'(x^0)=I
                $$
                $\Psi$ подходит под наш предыдущий случай, а значит мы нашли для него $N$:
                $$
                \mu\Psi(\Delta_N)\leqslant(1+2\eps)^n\mu\Delta_N
                $$
                При этом
                $$
                \mu\Psi(\Delta_N)=\mu S(\Phi(\Delta_N))=|\det S|\mu \Phi(\Delta_N)=\frac1{|\det\Phi'(x^0)|}\mu\Phi(\Delta_N)
                $$
                Из двух этих утверждений
                $$
                \mu\Phi(\Delta_N)\leqslant(1+2\eps^n)|\det\Phi'(x^0)|\mu\Delta_N
                $$
                А ещё мы знаем, что $C\mu\Delta_n\leqslant\mu\Phi(\Delta_n)$. А это уже капец:
                $$
                C<(1+2\eps)^n|\det\Phi'(x^0)|\longrightarrow C\leqslant|\det\Phi(x^0)|
                $$
                А у нас по выбору $C$ $C>\sup\limits_{\Delta}|\det\Phi'|\overset{|\det\Phi'|\text{непрерывно}}=\sup\limits_{\overline\Delta}|\det\Phi'|>|\det\Phi'(x^0)|$
            \end{enumerate}
            \item Пусть $E=U$ открытое подмножество $G$. Открытое множество можно представить как счётное объединения ячеек:
            $$
            U=\bigsqcup\limits_k D_k
            $$
            Где $D_k$~--- кубические ячейки, $\overline{D_k}\subset U$. Тогда
            \[\begin{split}
                \mu\Phi(U)&=\mu\Phi\left(\bigsqcup\limits_k D_k\right)=\mu\bigsqcup\limits_k\Phi(D_k)=\sum\limits_k\mu\Phi(D_k)\leqslant\sum\limits_k\mu D_k\sup\limits_{D_k}|\det\Phi'|\leqslant\\
                &\leqslant\sum\limits_k\mu D_k\sup\limits_{U}|\det\Phi'|=\sup\limits_U|\det\Phi'|\sum\limits_k\mu D_k=\mu U\sup\limits_U|\det\Phi'|
            \end{split}\]
            \item От открытого множества у произвольному $E\in\mathbb A_n(G)$. Тут будем пользоваться регулярностью меры Лебега:
            $$
            \mu\Phi(E)=\inf\limits_{\substack{V\text{ открыто}\\\Phi(E)\subset V\subset\Phi(G)}}\mu V=\inf\limits_{\substack{U\text{ открыто}\\E\subset U\subset G}}\mu\Phi(U)\leqslant\inf\limits_{\substack{U\text{ открыто}\\E\subset U\subset G}}\left(\mu U\sup\limits_U|\det\Phi'|\right)
            $$
            Хочется доказать, что это равно $\mu E\sup\limits_E|\det\Phi'|$. Мы знаем, что правая часть больше либо равна $\mu E\sup\limits_E|\det\Phi'|$, а нам надо доказать, что меньше либо равно.\\
            Если $\mu E=0$, то неравенство очевидно (тогда $\mu\Phi(E)=0$, гладкое отображение переводит множество меры ноль в множество меры ноль). Если $\mu E=+\infty$, nо доказывать нечего. И если супремум $\sup\limits_E|\det\Phi'|$ равен $+\infty$, то тоже (в правой части либо 0 (когда $\mu E=0$), либо бесконечность; в обоих случаях доказывать нечего).\\
            Далее считаем $\mu E,\sup\limits_E|\det\Phi'|\in(0;+\infty)$. Возьмём $\eps>0$.
            $$
            \exists\text{октрытое }U,E\subset U\subset G,\mu U\leqslant\mu E+\eps,\sup\limits_U|\det\Phi'|\leqslant\sup\limits_E|\det\Phi'|+\eps
            $$
            первое условие~--- регулярность меры Лебега, а второе вот:
            $$
            x\in E~\exists V_x\subset G~\forall t\in V_x~\left||\det\Phi'(t)|-|\det\Phi'(x)|\right|\leqslant\eps
            $$
            Тогда
            $$
            U=\bigcup\limits_{x\in E}V_x
            $$
            Такое $U$ подходит под
            $$
            \sup\limits_U|\det\Phi'|\leqslant\sup\limits_E|\det\Phi'|+\eps
            $$
            Если пересечь его с тем, которое в регулярности меры Лебега, то получится искомое $U$ из утверждения выше. Тогда
            $$
            \inf\limits_{\substack{U\text{ открыто}\\E\subset U\subset G}}\left(\mu U\sup\limits_U|\det\Phi'|\right)\leqslant (\mu E+\eps)\left(\sup\limits_E|\det\Phi'|+\eps\right)
            $$
            Устремив $\eps$ к нулю, получим искомое.
        \end{enumerate}
    \end{proof}
    \begin{theorem}[Замена переменной в кратном интеграле]
        \label{Замена переменной в кратном интеграле}
        Пусть $G\subset\mathbb R^n$ открыто, $\Phi$~--- диффеоморфизм на $G$, $E\in\mathbb A_n(G)$, $f\in S(\Phi(E))$. Тогда
        $$
        \int_{\Phi(E)}f~\mathrm d\mu=\int_E(f\circ\Phi)|\det\Phi'|~\mathrm d\mu
        $$
        Интегралы существуют или не существуют одновременно, если существуют, то равны.\\
        Также равенство пишется как
        $$
        \int_{\Phi(E)}f(y)~\mathrm dy=\int_E(f\circ\Phi)(x)|\det\Phi'(x)|~\mathrm dx
        $$
    \end{theorem}
    \begin{proof}
        Возьмём теорему \ref{Общая схема замены переменной в интеграле} и возьмём в ней
        $$
        (X;\mathbb A;\mu)=(G;\mathbb A_n(G);\mu)\qquad (Y;\mathbb B;\nu)=(\Phi(G);\mathbb A_N(\Phi(G));\mu)\qquad h=|\det\Phi'|
        $$
        От нас хотят равенство
        $$
        \forall B\in\mathbb B~\Phi^{-1}(B)\in\mathbb A,\nu B=\int\limits_{\Phi^{-1}(B)}\int h~\mathrm d\mu
        $$
        Ну, если $B=\Phi(E)$, то $\nu\Phi(E)=\int_E|\det\Phi'|~\mathrm d\mu$. А это мы доказали в \ref{Преобразование меры Лебега при диффеоморфизме}.
    \end{proof}
    \begin{corollary}
        В условии теоремы \ref{Замена переменной в кратном интеграле}
        $$
        f\in L(\Phi(E))\Leftrightarrow(f\circ\Phi)|\det\Phi'|\in L(E)
        $$
    \end{corollary}
    \begin{corollary}
        Пусть $G\subset H\subset\mathbb R^n$, $G$ открыто. Пусть $\Phi\colon H\to\mathbb R^n$ и пусть $\Phi\big|_G$~--- диффеоморфизм. И пусть ещё $\mu(H\setminus G)=\mu(\Phi(H)\setminus\Phi(G))=0$. $E\in\mathbb A_n(H)$, $f\in S(\Phi(E))$. Тогда верна формула замены переменной:
        $$
        \int_{\Phi(E)}f(y)~\mathrm dy=\int_E(f\circ\Phi)(x)|\det\Phi'(x)|~\mathrm dx
        $$
    \end{corollary}
    \begin{proof}
        $$
        \int_{\Phi(E)}f(y)~\mathrm dy=\int_{\Phi(E\cap G)}f(y)~\mathrm dy=\int_{E\cap G}(f\circ\Phi)(x)|\det\Phi'(x)|~\mathrm dx=\int_E(f\circ\Phi)(x)|\det\Phi'(x)|~\mathrm dx
        $$
    \end{proof}
    \begin{remark}
        Ослаблять условия теоремы \ref{Замена переменной в кратном интеграле} дальше трудно и больно. Но можно. Но туда мы лезть не будем. Для желающих есть книжка Эванса и Гариепи <<Теория меры и тонкие свойства функций>>.
    \end{remark}
    \begin{remark}
        А что у нас в $n=1$, как это коррелирует с тем, что мы знаем?
        $$
        \int\limits_a^bf=\int\limits_\alpha^\beta(f\circ\phi)\phi'\qquad\phi(\alpha)=a,\phi(\beta)=b
        $$
        Почему модуль? На самом деле у нас и тут есть модуль? Потому что $\phi$ может как возрастать, так и убывать, и во втором случае у нас меняются местами пределы интегрирования. А в формуле \ref{Замена переменной в кратном интеграле} ориентации на $E$ не задано.
    \end{remark}
    \begin{example}
        Сдвиг и отражение.\\
        $\Phi(x)=a\pm x$, $a\in\mathbb R^n$. Очевидно, это диффеоморфизм и модуль якобиана равен единице. То есть
        $$
        \int_{\mathbb R^n}f(y)~\mathrm dy=\int_{\mathbb R^n}f(a\pm x)~\mathrm dx
        $$
    \end{example}
    \begin{example}
        Полярные координаты в $\mathbb R^2$.\\
        $(x;y)=(r\cos\varphi;r\sin\varphi)=\Phi(r;\varphi)$. Посчитаем якобиан:
        $$
        \matrd{\cos\varphi & -r\sin\varphi\\\sin\varphi & r\cos\varphi}=r
        $$
        Это не равно нулю всюду, кроме начала координат. Где $\Phi$~--- диффеоморфизм? По-разному можно отвечать, например, так: удалим из плоскости отрицательную часть вещественной оси. Тогда $G=(0;+\infty)\times(-\pi;\pi)$, $\Phi(G)=\mathbb R^2\setminus\{(x;0)\mid x\leqslant0\}$. Тогда $\Phi$~--- диффеоморфизм. Очевидно, пренебрегать тем, что мы сделали, можно, там мера ноль.\\
        Посчитаем следующий интеграл:
        $$
        I=\int\limits_0^\infty e^{-x^2}~\mathrm dx
        $$
        Это не берётся, но посчитать можно:
        \[\begin{split}
            I^2&=\left(\int_0^\infty e^{-x^2}~\mathrm dx\right)\left(\int_0^\infty e^{-y^2}~\mathrm dy\right)=\iint_{(0;+\infty)^2}e^{-x^2-y^2}~\mathrm dx\mathrm dy\overset{\substack{x=r\cos\varphi\\y=r\sin\varphi}}=\\
            &=\int\limits_0^{+\infty}\int\limits_0^{\frac\pi2}e^{-r^2}r~\mathrm d\varphi\mathrm dr=\frac\pi2\int_0^{+\infty}e^{-r^2}r~\mathrm dr=\frac\pi2\frac{-e^{-r^2}}2\bigg|_{0}^{+\infty}=\frac\pi4
        \end{split}\]
        Отсюда $I=\frac{\sqrt\pi}2$
    \end{example}
    \begin{example}
        Цилиндрические координаты: $(x;y;z)=(r\cos\varphi;r\sin\varphi;h)=\Phi(r;\varphi;h)$. Тогда $\det\Phi'=r$, $G=(0;+\infty)\times(-\pi;\pi)\times\mathbb R$, $\Phi(G)=\{(x;0;z)\mid x\leqslant 0,z\in\mathbb R\}$.
    \end{example}
    \begin{example}
        Сферические координаты: %TODO: pic
        $$
        \left\{\begin{aligned}
            \rho&=r\cos\psi\\
            z&=r\sin\psi
        \end{aligned}\right.\qquad
        \left\{\begin{aligned}
            x&=\rho\cos\varphi\\
            y&=\rho\sin\varphi
        \end{aligned}\right.
        $$
        (Обозначения могут быть другими.)\\
        Тогда
        $$
        (x;y;z)=(r\cos\varphi\cos\psi;r\sin\varphi\cos\psi;r\sin\psi)=\Phi(r;\varphi;\psi)
        $$
        Из написанного выше $\Phi$ можно представить как два полярных преобразования, а якобиан произведения равен произведению якобианов, т.е. $\det\Phi'=r\rho=r^2\cos\psi$.\\
        Что покусать из пространства? Ну,
        $$
        G=(0;+\infty)\times(-\pi;\pi)\times\left(-\frac\pi2;\frac\pi2\right)\qquad\Phi(G)=\{(x;0;z)\mid x\leqslant 0,z\in\mathbb R\}
        $$
    \end{example}
    \begin{example}
        Сферические координаты в $\mathbb R^n$.\\
        $x\in\mathbb R^n$, $r\in(0;+\infty)$, $\varphi\in\mathbb R^{n-1}$. Тут тоже последовательные полярные замены:
        $$
        \left\{\begin{aligned}
            x_1&=\rho_1\cos\varphi_1\\
            x_2&=\rho_1\sin\varphi_1
        \end{aligned}\right.\qquad
        \left\{\begin{aligned}
            \rho_1&=\rho_2\cos\varphi_2\\
            x_3&=\rho_2\sin\varphi_2
        \end{aligned}\right.\qquad\cdots\qquad
        \left\{\begin{aligned}
            \rho_{n-3}&=\rho_{n-2}\cos\varphi_{n-2}\\
            x_{n-1}&=\rho_{n-2}\sin\varphi_{n-2}
        \end{aligned}\right.\qquad
        \left\{\begin{aligned}
            \rho_{n-2}&=r\cos\varphi_{n-1}\\
            x_n&=r\sin\varphi_{n-1}
        \end{aligned}\right.
        $$
        Отсюда $$
        \det\Phi'=\rho_1\rho_2\cdots\rho_{n-2}r=r^{n-1}\cos^{n-2}\varphi_{n-1}\cos^{n-1}\varphi_{n-2}\cdots\cos^2\varphi_3\cos\varphi_2
        $$
        $$
        \left\{\begin{aligned}
            x_1&=r\cos\varphi_{n-1}\cos\varphi_{n-2}\cdots\cos\varphi_2\cos\varphi_1\\
            x_2&=r\cos\varphi_{n-1}\cos\varphi_{n-2}\cdots\cos\varphi_2\sin\varphi_1\\
            x_3&=r\cos\varphi_{n-1}\cos\varphi_{n-2}\cdots\sin\varphi_2\\
            \vdots&\\
            x_{n-1}&=r\cos\varphi_{n-1}\sin\varphi_{n-2}\\
            x_n&=r\sin\varphi_{n-1}
        \end{aligned}\right.
        $$
        В качестве $G$ можно берётся вот что:
        $$
        G=(0;+\infty)\times(-\pi;\pi)\times\left(-\frac\pi2;\frac\pi2\right)^{n-2}
        $$
        Желающие узнать $\Phi(G)$, могут почитать учебник.
    \end{example}
    \paragraph{Мера и интеграл Лебега~--- Стилтьеса.}
    \begin{definition}
        Пусть $\Delta=(\alpha;\beta)\subset\mathbb R$. Пусть $g\colon\Delta\to\mathbb R$ возрастает и непрерывна слева. Пусть $\mathbb P_\Delta$~--- множество ячеек (полуинтервалов), содержащихся в $\Delta$ вместе с замыканием:
        $$
        \mathbb P_\Delta=\{[a;b)\mid \alpha<a\leqslant b<\beta\}
        $$
        (В случае $\Delta=\mathbb R$ $\mathbb P_\Delta=\mathbb P_1$.) Очевидно, $\mathbb P_\Delta$~--- полукольцо.\\
        \textbf{Объём, порождённый функцией} $g$~--- $V_g[a;b)=g(b)-g(a)$.
    \end{definition}
    \begin{property}
        Очевидно, это объём.
    \end{property}
    \begin{property}
        Объём, порождённый функцией~--- мера.
    \end{property}
    \begin{proof}
        См. доказательство для меры Лебега, но используя следующее равенство:
        $$
        \lim\limits_{n\to\infty}V_g\left[a-\frac1n;b\right)=Vg[a;b)=\lim\limits_{n\to\infty}V_g\left[a;b-\frac1n\right)
        $$
    \end{proof}
    \begin{definition}
        Стандартное продолжение $V_g$ на некоторую $\sigma$-алгебру называется \textbf{мерой Стилтьеса~--- Лебега}, порождённой функцией $g$ (и обозначается $\mu_g$).\\
        Сигма-алгебру, на которой определена эта мера, обозначают $\mathbb A_g$.
    \end{definition}
    \begin{remark}
        Мера Лебега $\mu_1$ является частным случаем меры Стилтьеса~--- Лебега при $g(x)=x$, $\Delta=\mathbb R$.
    \end{remark}
    \begin{property}
        Мера одноточечного множества $\{a\}$ равна $g(a+)-g(a)$
    \end{property}
    \begin{proof}
        $$
        \{a\}=\bigcap\limits_{n=1}^\infty \left[a;a+\frac1n\right)
        $$
        По непрерывности меры
        $$
        \mu_g\{a\}=\lim\limits_{n\to\infty}\mu_g\left[a;a+\frac1n\right)=\lim\limits_{n\to\infty}g\left(a+\frac1n\right)-g(a)=g(a+)-g(a)
        $$
    \end{proof}
    \begin{property}
        Аналогично
        \begin{align*}
            \mu_g[a;b]&=g(b+)-g(a)\\
            \mu_g(\alpha;b]&=g(b+)-g(\alpha+)\\
            \mu_g[a;\beta)&=g(\beta-)-g(a)\\
            \mu_g(\alpha;\beta)&=g(\beta-)-g(\alpha+)
        \end{align*}
    \end{property}
    \begin{remark}
        Мера точки может быть положительной. Нулю она равна тогда и только тогда, когда $g$ непрерывна в этой точке.
    \end{remark}
    \begin{property}
        Мера Лебега~--- Стилтьеса $\sigma$-конечна. Конечна она тогда и только тогда, когда $\mu_g$ ограничена.
    \end{property}
    \begin{definition}
        Определим меру $\mu_g$ на промежутке произвольного типа.\\
        Пусть $\Delta=\langle\alpha;\beta\rangle\subset\mathbb R$.\\
        Если $\alpha\in\Delta$, то пусть $\widetilde g(x)=g(\alpha)$ при $x<\alpha$.\\
        Если $\beta\in\Delta$, то не требуем непрерывности $g$ слева в точке $\beta$, но положим
        $$\widetilde g(x)=\begin{cases}
            g(x) & x\in\langle\alpha;\beta)\\
            g(\beta-) & x=\beta\\
            g(\beta) & x>\beta
        \end{cases}$$
        После этого получим $\widetilde g$, заданное на открытом промежутке $\widetilde\Delta\supset\Delta$. При этом на нём $\widetilde g$ возрастает и непрерывно слева. Положим, что мера Лебега~--- Стилтьеса $\mu_g$ равна $\mu_{\widetilde g}\big|_{\mathbb A_{\widetilde g}(\Delta)}$.
    \end{definition}
    \begin{remark}
        Также можно определить меру $\mu_g$ для функции $g$, которая возрастает на $\Delta$, но не обязательно непрерывна слева. Тогда мы просто исправляем $g$ в точках левого разрыва (кроме $\beta$).\\
        Другой способ~--- просто определить меру Лебега~--- Стилтьеса как $\mu_g[a;b)=g(b-)-g(a-)$.
    \end{remark}
    \begin{remark}
        У этих мер есть проблемы: $\mathbb A_g$ различны для разных $g$. А иногда хочется сложить две меры Лебега~--- Стилтьеса. Тогда их сужают на Борелевскую $\sigma$-алгебру (на ней они все определены т.к. определены на ячейках), получая меру Бореля~--- Стилтьеса (на самом деле бывает более широкая сигма-алгебра, но обычно хватает Борелевской).
    \end{remark}
    \begin{lemma}
        Пусть $\Delta$~--- промежуток в $\mathbb R$, есть мера $\nu$, заданная на $\mathbb B_\Delta$, которая конечна на $\mathbb P_\Delta$ (ячейках, лежащих в $\Delta$ вместе с замыканием). Тогда существует такая $g\uparrow\Delta$, что $\nu=\mu_g\big|_{\mathbb B_\Delta}$.
    \end{lemma}
    \begin{proof}
        Пусть для определённости $\Delta$ открыт.\\
        Пусть $x\in\Delta$. Определим $g$ так:
        $$
        g(x)=\begin{cases}
            \nu[x_0;x) & x\geqslant x_0\\
            -\nu[x;x_0) & x<x_0
        \end{cases}
        $$
        Возрастание $g$ на $\Delta$ очевидно. Проверим непрерывность слева. Пусть для определённости $x>x_0$. Рассмотрим $u\in(x_0;x)$. Тогда
        $$
        g(u)=\nu[x_0;u)\underset{u\to x-}\longrightarrow\nu[x_0;x)=g(x)
        $$
        Если же $x\leqslant x_0$, возьмём $u<x$:
        $$
        g(u)=-\nu[u;x_0)\underset{u\to x-}\longrightarrow-\nu[x;x_0)=g(x)
        $$
        (Здесь мы используем конечность на отрезках.)\\
        Осталось проверить, что $\nu$ и $\mu_g$ совпадают на ячейках. Рассмотрим $[a;b)\subset[a;b]\subset\Delta$. Тогда
        $$
        \mu_g[a;b)=g(b)-g(a)=\begin{cases}
            \nu[x_0;b)-\nu[x_0;a) & x_0\leqslant a<b\\
            \nu[x_0;b)-(-\nu[a;x_0)) & a<x_0<b\\
            -\nu[b;x_0)-(-\nu[a;x_0)) & a<b\leqslant x_0
        \end{cases}=\nu[a;b)
        $$
    \end{proof}
    \begin{definition}
        \textbf{Интегралом Лебега~--- Стилтьеса} называется никогда не догадаетесь что. Помимо стандартного обозначения $\int_Ef~\mathrm d\mu_g$ также пишут
        $$
        \int_Ef~\mathrm dg\qquad \int_Ef(x)~\mathrm dg(x)
        $$
        $f$ называутся \textbf{интегрируемой функцией} (integrant), а $g$~--- \textbf{интегрирующей функцией} (integrand).
    \end{definition}
    \begin{claim}
        Пусть $(X;\mathbb A;\mu)$~--- пространство с мерой, а $\mathbb B\subset\mathbb A$. Пусть $f\in S_{\mathbb B}(X)$. Тогда
        $$
        \int_Xf~\mathrm d\mu=\int_Xf~\mathrm d\mu\big|_{\mathbb B}
        $$
    \end{claim}
    \begin{proof}
        Частный случай \hyperref[Общая схема замены переменной в интеграле]{замены переменной в интеграле}, для $\Phi=\mathrm{id}_X$.
    \end{proof}
    \begin{remark}
        Дальше рассмотрим несколько частных случаев меры и интеграла Лебега~--- Стилтьеса.
    \end{remark}
    \begin{example}
        Дискретная мера.\\
        Введём
        $$\theta(x)=\begin{cases}
            1 & x>0\\0 & x\leqslant 0
        \end{cases}$$
        --- тета-функцию Хевисайда. Очевидно, что
        $$\mu_\theta E=\delta_0E=\begin{cases}
            1 & 0\in E\\
            0 & 0\notin E
        \end{cases}$$
        А дальше рассмотрим $\{a_k\}$~--- не более чем счётный набор точек из $\Delta$, $\Delta$ открыт в $\mathbb R$, $\{h_k\}\subset(0;+\infty)$. (Дальше будем считать, что имеем счётный набор, конечный будет частным случаем (много нулей)). Пусть
        $$
        \forall [a;b]\subset\Delta~\sum\limits_{k:a_k\in[a;b)}h_k<+\infty
        $$
        Тогда возьмём $x\in\Delta,c\in\mathbb R$ и определим
        $$
        g(x)=c+\sum\limits_kh_k(\theta(x-a_k)-\theta(x_0-a_k))
        $$
        Заметим, что $\theta$ возрастает, а значит при фиксированном $x$ все слагаемые одного знака. Поэтому сумма ряда есть в $\overline{\mathbb R}$. На самом деле эта сумма конечна.\\
        Пусть $x\geqslant x_0$. Тогда, выкинув из суммы нулевые слагаемые, получим
        $$
        c\leqslant g(x)=c+\sum\limits_{k:a_k\in[x_0;x)}h_k<+\infty
        $$
        В случае $x<x_0$ аналогично. А ещё $g$ возрастает (т.к. это сумма ряда возрастающих функций).
        \begin{claim}
            $g$ непрерывна везде, кроме $a_k$, а в $a_k$ непрерывна только слева, а скачок равен $h_k$.
        \end{claim}
        \begin{proof}
            Докажем, что ряд в определении $g$ сходится равномерно на любом отрезке, содержащемся в $\Delta$.\\
            Достаточно доказывать для $[a;x_0]$ и $[x_0;b]$ (остальные являют собой объединение или разность двух таких). Рассмотрим $[x_0;b]$, $[a;x_0]$ аналогично.\\
            Рассмотрим $x\in[x_0;b]$. Заметим, что
            $$
            h_k(\theta(x-a_k)-\theta(x_0-a_k))\leqslant h_k(\theta(b-a_k)-\theta(x_0-a_k))
            $$
            А ряд с тем, что справа, сходится (т.к. поточечно сходится $g(b)$), то есть $g$ равномерно сходится по признаку Вейерштрасса.\\
            Заметим, что все члены ряда непрерывны на $\Delta\setminus\{a_k\}_k$, а значит и сумма тоже. Также заметим, что если удалить из суммы $k$-тое слагаемое, то всё остальное будет непрерывным. А $h_k(\theta(x-a_k)-\theta(x_0-a_k))$ ведёт себя так, как нам хочется.
        \end{proof}
    \end{example}
    \begin{definition}
        Функция $g$ такого вида, как в примере выше, называется \textbf{функцией скачков}.
    \end{definition}
    \begin{theorem}[Дискретная мера как мера Лебега~--- Стилтьеса]
        В условиях определения $g$, $\mathbb A_g=2^\Delta$, $\mu_g$~--- дискретная мера с нагрузками $h_k$ в точках $a_k$.
    \end{theorem}
    \begin{proof}
        Если $[a;b]\subset\Delta$, то
        $$
        \mu_g[a;b)=g(b)-g(a)=\sum\limits_kh_k(\theta(b-a_k)-\theta(a-a_k))=\sum\limits_{k:a_k\in[a;b)}h_k
        $$
        Кажется, это ровно определение дискретной меры. Обозначим её за $\nu[a;b)$. Кайф, две меры совпадают на ячейках. А значит совпадают на $\mathbb A_g$. (В книжке есть доказательство без теоремы о единственности стандартного продолжения меры.) Остаётся лишь доказать, что $\mathbb A_g=2^\Delta$. Рассмотрим
        $$H=\{a_1;a_2;\ldots\}$$
        Это множество измеримо (как не более чем счётное). Значит и дополнение его измеримо. А $\mu_g(\Delta\setminus H)=0$. А меры $\mu_g$ (как и любое стандартное продолжение) полна, а значит любое подмножество $\Delta\setminus H$ измеримо (и имеет меру ноль). Ну и всё, $A=(A\cap H)\cup(A\cap H\compl)$, и первое измеримо как не более чем счётное, а второе как множество меры ноль.
    \end{proof}
    \begin{remark}
        Как мы видим, $\mu_g$ не зависит от $x_0$ и $c$. А значит, если ряд $\sum\limits_kh_k \theta(x-a_k)$ сходится для любого $x$, то $\sum\limits_kh_k \theta(x_0-a_k)$ можно вынести в $c$.
    \end{remark}
    \begin{example}
        $g(x)=\lfloor x\rfloor$ порождает считающую меру на $\mathbb Z$.
    \end{example}
    \begin{remark}
        Если $\alpha\in\Delta$ или $\beta\in\Delta$, то можно добавить нагрузки в этих точках.
    \end{remark}
    \begin{corollary}
        $$
        \int_Ef~\mathrm dg=\sum\limits_{k:a_k\in E}f(a_k)h_k
        $$
    \end{corollary}
    \begin{remark}
        Мы видим $\mathrm dg$. Очень хочется заменить это на $g'\mathrm dx$. В рассмотренном выше примере это не получится без обобщённых функций, но в некоторых примерах получится.
    \end{remark}
    \begin{remark}
        Здесь и далее $\int\limits_a^bf$~--- интеграл Лебега, $\int\limits_b^af=-\int\limits_a^bf$.
    \end{remark}
    \begin{definition}
        Пусть $\Delta$~--- промежуток в $\mathbb R$, $h\colon\Delta\to\overline{\mathbb R}$ называется \textbf{локально суммируемой} на $\Delta$, еси она суммируема на любом отрезке в $\Delta$. Обозначение: $L_{\mathrm{loc}}(\Delta)$.
    \end{definition}
    \begin{definition}
        Пусть $\Delta$~--- промежуток в $\mathbb R$. $g\colon\Delta\to\mathbb R$ называется \textbf{локально абсолютно непрерывной} на $\Delta$, если $g$ представляется в виде $g(x)=\int_{x_0}^xh+\underbrace{\mathrm{const}}_{g(x_0)}$, где $x_0\in\Delta$, $h$ локально суммируема. Обозначение $AC_{\mathrm{loc}}(\Delta)$.
    \end{definition}
    \begin{remark}
        Далее мы будем опускать слово <<локально>> в этом термине.
    \end{remark}
    \begin{property}
        Если $h$ непрерывно в точке $x$, то $g$ дифференцируема в ней и $g'(x)=h(x)$.
    \end{property}
    \begin{proof}
        См. доказательство теоремы Барроу.
    \end{proof}
    \begin{property}
        По теореме Барроу и формуле Ньютона~--- Лейбница
        $$
        C^{(1)}(\Delta)\subsetneqq AC_{\mathrm{loc}}(\Delta)
        $$
    \end{property}
    \begin{proof}
        Включение строгое, если в качестве $h$ взять $\theta$.
    \end{proof}
    \begin{property}
        $$
        AC_{\mathrm{loc}}(\Delta)\subsetneqq C(\Delta)
        $$
    \end{property}
    \begin{property}
        Для включения см. теорему об абсолютной непрерывности интеграла из прошлого семестра.\\
        Включение строгое:
        $$
        g(x)=\int_x^1\frac1t\sin\frac1t~\mathrm dt\qquad \Delta=[0;1]
        $$
        В $x=0$ условно сходится (а значит функция $g$ непрерывна), но абсолюно непрерывной она не будет т.к. единственный кандидат на роль $h$ ($\frac1t\sin\frac1t$) не суммируемо.
    \end{property}
    \begin{remark}
        Более интересно то, что включение строгое, даже если рассматривать только возрастающие функции. Например, если взять $\Theta$~--- канторову лестницу, то $h$ по крайней мере в дополнительных канторовых промежутках была равна производной $\Theta$, а значит почти везде была бы равна нулю.
    \end{remark}
    \begin{property}
        Если $g\in AC_{\mathrm{loc}}(\Delta)$, то $g$ дифференцируема в почти всех точках $\Delta$, $g'\in L_{\mathrm{loc}}(\Delta)$ и $g'=h$ почти везде.\\
        Без доказательства.
    \end{property}
    \begin{corollary}
        Тогда
        $$
        g(x)=\int_{x_0}^xg'+g(x_0)
        $$
    \end{corollary}
    \begin{claim}
        Однако условие <<$g$ почти везде дифференцируема на $\Delta$ и $'g\in L_{\mathrm{loc}}(\Delta)$>> не влечёт равенства
        $$
        g(x)=\int_{x_0}^xg'+g(x_0)
        $$
        (и не влечёт абсолютной локальной непрерывности).
    \end{claim}
    \begin{proof}
        Канторова лестница.
    \end{proof}
    \begin{remark}
        Абсолютно локально непрерывные функции~--- в точности те функции, для которых верна формула Ньютона~--- Лейбница.
    \end{remark}
    \begin{claim}
        Если $g$ возрастает на $[a;b]$, то $g$ дифференцируема почти везде на $[a;b]$. Тогда $g'$ неотрицательно почти везде и
        $$
        \int\limits_a^bg'\leqslant g(b)-g(a)
        $$
        Без доказательства.
    \end{claim}
    \begin{theorem}[Интеграл Лебега~--- Стилтьеса абсолютно локально непрерывной функции.]
        \label{Интеграл Лебега~--- Стилтьеса абсолютно локально непрерывной функции}
        Пусть $\Delta$~--- промежуток в $\mathbb R$, $h\in L_{\mathrm{loc}}(\Delta)$, $h\geqslant0$,$x_0\in\Delta$,
        $$
        g(x)=\int_{x_0}^xh+g(x_0)\qquad x\in\Delta
        $$
        Тогда
        \begin{enumerate}
            \item $\mathbb A_1(\Delta)\subset\mathbb A_g(\Delta)$.
            \item Если $E\in\mathbb A_1(\Delta)$, $f\in S(E)$, то $\int_Ef~\mathrm dg=\int_Efh$. Интегралы существуют или нет одновременно, если существуют, то равны.
        \end{enumerate}
    \end{theorem}
    \begin{proof}
        Пусть $\nu E=\int_Eh$. Это мера на $\mathbb A_1(\Delta)$. Заметим, что тогда второе утверждение~--- \hyperref[Общая схема замены переменной в интеграле]{замена переменной в интеграле}. О'кей, заметим, что $\mu_g$ и $\nu$ равны нулю на одноточечном множестве, а значит можно считать $\Delta$ открытым.\\
        Круть, заметим, что $\mu_g[a;b)=g(b)-g(a)=\int\limits_a^bh=\nu[a;b)$, то есть $\nu$ и $\mu_g$ совпадают на ячейках. А значит совпадают на $\mathbb B_\Delta$. А хочется, чтобы они совпадали на измеримых Лебегу множествах.\\
        Хорошо, давайте дальше проверим, что $\nu$ и $\mu_g$ совпадают на множествах нулевой меры. Возьмём $e\subset\Delta:\mu_1e=0$. Его можно заключить в множестве типа $G_\delta$ с нулевой мерой, а на множестве типа $G_\delta$ $\nu$ и $\mu_g$ совпадают (и равны нулю). Тогда по полноте $\mu_g$ $e\in\mathbb A_g(\Delta)$.\\
        Итого рассмотрев множество $E\in\mathbb A_1(\Delta)$, представим его как $A\cup e$, где $A\in\mathbb B_\Delta$, $\mu_1e=0$, получим, что любое такое $E$ лежит в $\mathbb A_g(\Delta)$.\\
        Осталось применить теорему \ref{Общая схема замены переменной в интеграле} ($\Phi=\mathrm{id}_\Delta$).
    \end{proof}
    \begin{corollary}
        Если $g\in C^{(1)}(\Delta)$ и возрастает, то $\int_Ef~\mathrm dg=\int_Efg'$
    \end{corollary}
    \begin{remark}
        Очень жаль, но наши два примера~--- это не все функции.
    \end{remark}
    \begin{definition}
        $g\colon\Delta\to\mathbb R$ называется \textbf{сингулярной}, если $g\equiv0$ \textbf{или} $g$ непрерывна, $g\neq\mathrm{const}$ и $g'=0$ почти везде.
    \end{definition}
    \begin{example}
        Канторова лестница.
    \end{example}
    \begin{claim}
        Пусть $g\colon\Delta\to\mathbb R$, возрастает и непрерывна слева (кроме, возможно, правого конца $\Delta$). Тогда $g$ единственным образом (с точностью до константы) представляется в виде
        $$
        g=g_{\mathrm{disc}}+g_{\mathrm{c}}
        $$
        где первое~--- функция скачков, второе~--- непрерывная. При этом
        $$
        g_{\mathrm{c}}=g_{\mathrm{ac}}+g_{\mathrm{sing}}
        $$
        Где первое~--- абсолютно непрерывна, второе~--- сингулярна. При этом все эти $g$ также возрастают и непрерывны слева (кроме, возможно, левого конца).\\
        Что интересно, то же самое можно записать для мер Стелтьеса~--- Лебега (по крайней мере на $\mathbb B_\Delta$):
        $$
        \mu_g=\mu_{g_{\mathrm{disc}}}+\mu_{g_{\mathrm{ac}}}+\mu_{g_{\mathrm{sing}}}
        $$
    \end{claim}
    \begin{definition}
        \textbf{Интеграл Лебега~--- Стилтьеса функции произвольного знака}.\\
        Пусть $g=g_1-g_2$, где $g_1,g_2$ возрастают. Пусть $f,E$~--- борелевские. Тогда положим
        $$
        \int_Ef~\mathrm dg=\int_Ef~\mathrm dg_1-\int_Ef~\mathrm dg_2
        $$
        Если правая часть существует.
    \end{definition}
    \begin{property}
        Нетрудно заметить, что этот интеграл не зависит от конкретного разбиения $g$ на $g_1$ и $g_2$.
    \end{property}
    \begin{remark}
        В частности, на отрезке можно интегрировать по функции ограниченной вариации.
    \end{remark}
    \begin{property}
        Интеграл заведомо существует и конечен для борелевской ограниченной функции $f$.
    \end{property}
    \begin{property}
        Для таких интегралов верна теорема \ref{Интеграл Лебега~--- Стилтьеса абсолютно локально непрерывной функции}.
    \end{property}
    \begin{theorem}[Интегрирование по частям в интеграле Лебега~--- Стилтьеса.]
        \label{Интегрирование по частям в интеграле Лебега~--- Стилтьеса}
        Пусть $f\in AC[a;b]$, $g\in V[a;b]$. Тогда
        $$
        \int_{[a;b]} f~\mathrm g=fg\bigg|_a^b-\int_a^bf'g
        $$
    \end{theorem}
    \begin{proof}
        Считаем, что $g$ возрастает, (иначе представим в виде разности двух возрастающих) и непрерывна слева (кроме, может, $b$).
        \begin{itemize}
            \item Докажем сначала формулу в частном случае $f(a)=g(b)=0$. Тогда
            $$
            \int_{[a;b]}f~\mathrm dg=\int_{[a;b]}\left(\int\limits_a^xf'(u)~\mathrm du\right)~\mathrm dg(x)
            $$
            Хочется воспользоваться \hyperref[Теорема Фубини]{теоремой Фубини}. Тогда заметим, что $a\leqslant u\leqslant x\leqslant b$, то есть имеем треугольник. Чтобы менять порядок интегрирования, надо проверить суммируемость подынтегральной функции. Для этого ставим модуль $|f'(u)|$. Тогда изменить порядок интегрирования можно по \hyperref[Теорема Тонелли]{Тонелли} и получить $\int\limits_a^b|f'|g$, где первое суммируемо, второе ограничено, а значит интеграл небесконечен. То есть $f'(u)$ суммируема
            $$
            \int_{[a;b]}f~\mathrm dg=\int_{[a;b]}f'(u)\underbrace{\left(\int_{[u;b]}\mathrm dg(x)\right)}_{\mu_g[u;b]=g(b)-g(u)=-g(u)}~\mathrm du=-\int\limits_a^bf'g
            $$
            Получили то, что хотели.
            \item Общий случай: рассмотрим $f-f(a)$ и $g-g(b)$. По доказанному,
            $$
            \int_{[a;b]}f-f(a)~\mathrm d(g-\bcancel{g(b)})=-\int\limits_a^b(f-\cancel{f(a)})'(g-g(b))
            $$
            Тогда
            $$
            \int_{[a;b]}f~\mathrm dg-f(a)\int_{[a;b]}\mathrm dg=-\int\limits_a^bf'g+\int\limits_a^bf'g(b)
            $$
            При этом $\int_{[a;b]}\mathrm dg$ мы уже считали, это $g(b)-g(a)$, а $\int\limits_a^bf'=f(b)-f(a)$ по формуле Ньютона~--- Лейбница ($f\in AC[a;b]$). Приведя подобные слагаемые, получим искомое.
        \end{itemize}
    \end{proof}
    \begin{corollary}[Интегрирование по частям в интеграле Лебега]
        \label{Интегрирование по частям в интеграле Лебега}
        Если $f,g\in AC[a;b]$, то
        $$
        \int\limits_a^bfg'=fg\bigg|_a^b-\int\limits_a^bf'g
        $$
    \end{corollary}
    \paragraph{Интегралы, зависящие от параметра.}
    \begin{definition}
        Пусть $(X;\mathbb A;\mu)$~--- пространство с мерой, $Y$~--- множество (произвольное). И есть функция $f\colon X\times Y\to\overline{\mathbb R}$. Пусть также
        $$\forall y\in Y~f(\bullet;y)\in L(X;\mu)$$
        Тогда $I(y)=\int_Xf(x;y)~\mathrm dx$ называется \textbf{интегралом, зависящим от параметра}.
    \end{definition}
    \begin{remark}
        Чтобы исследовать свойства интеграла с параметром, придётся вводить дополнительную структуру на $Y$. Например, если $Y$~--- метрическое пространство, можно ли перейти к пределу под знаком интеграла? Или есть ли непрерывность интеграла с параметром. Что можно сказать о дифференцируемости $I$ и о её производной (тогда уже надо считать $Y$ подмножеством $\mathbb R^n$). Можно ли интегрировать по $y$, если $Y$~--- пространство с мерой?\\
        Впрочем, ответ на 4 вопросы мы знаем~--- см. теоремы \hyperref[Теорема Тонелли]{Тонелли} и \hyperref[Теорема Фубини]{Фубини}. На остальные сейчас попытаемся ответить.
    \end{remark}
    \begin{theorem}[Предельный переход по параметру под знаком интеграла]
        \label{Предельный переход по параметру под знаком интеграла}
        Пусть $(X;\mathbb A;\mu)$~--- пространство с мерой, $\tilde Y$~--- пространство с морой, $Y\subset Y$. Пусть $f\colon X\times Y\to\overline{\mathbb R}$ и пусть
        $$\forall y\in Y~f(\bullet;y)\in L(X;\mu)$$
        Пусть $y_0$~--- предельная точка $Y$, при почти всех $x\in X$ $f(x;y)\underset{y\to y_0}\rightarrow g(x)$.\\
        Пусть
        $$\exists\Phi\in L(X;\mu)~\exists V_{y_0}~\text{при почти всех }x\in X~\forall y\in{\dot V}_{y_0}\cap Y~|f(x;y)|\leqslant\Phi(x)$$
        Тогда
        $$
        \lim\limits_{y\to y_0}\int_Xf(x;y)~\mathrm d\mu(x)=\int_X\lim\limits_{y\to y_0}f(x;y)~\mathrm d\mu(x)
        $$
    \end{theorem}
    \begin{definition}
        Условие
        $$\exists\Phi\in L(X;\mu)~\exists V_{y_0}~\text{при почти всех }x\in X~\forall y\in{\dot V}_{y_0}\cap Y~|f(x;y)|\leqslant\Phi(x)$$
        называется \textbf{локальным условием Лебега} в точке $y_0$.
    \end{definition}
    \begin{proof}
        Возьмём последовательность точек $y_n\in Y\setminus\{y_0\}$, $y_n\to y_0$. Тогда начиная с некоторого $y_N$ все $y_{n>N}\in V_{y_0}$ из локального условия Лебега.\\
        Введём последовательность функций $f_n(x)=f(x;y_n)$. Тогда при почти всех $x\in X$ $f_n(x)\underset{n\to\infty}\rightarrow g(x)$. Кроме того в силу локального условия Лебега для почти всех $x\in X~\forall n\in\mathbb N~|f_n(x)|\leqslant\Phi(x)$\\
        То теореме Лебега о мажорируемой сходимости $g\in L(X;\mu)$, и
        $$
        \int_Xf_n(x)~\mathrm d\mu(x)\underset{n\to\infty}\longrightarrow\int_Xg(x)~\mathrm d\mu(x)
        $$
    \end{proof}
    \begin{remark}
        Не исключён случай, когда $y_0$~--- бесконечно удалённая точка или $\pm\infty$, если $\tilde Y=\mathbb R$.
    \end{remark}
    \begin{remark}
        Квантор $\forall$ и <<для почти всех>> в общем случае менять нельзя. <<Почти всех $x~\forall y$>> сильнее, чем <<$\forall y$ для почти всех $x$>>. Но для данной теоремы более слабое условие также работает (без изменения доказательства).
    \end{remark}
    \begin{remark}
        Интересный факт: мы имели равномерную сходимость для рядов. А эта теорема в некотором смысле оперирует с равномерной сходимостью для семейств функций ($f$ можно рассматривать как семейство функций $f_y(x)$).
    \end{remark}
    \begin{definition}
        Пусть $X$~--- множество, $\tilde Y$~--- метрическое пространство, $Y\subset\tilde Y$, $y_0$~--- передельная точка $Y$, $f\colon X\times Y\to\mathbb R$ (или $\mathbb C$), $g\colon X\to\mathbb R$ (или $\mathbb C$). Тогда говорят, что \textbf{семейство функций} $\{f(\bullet;y)\}_{y\in Y}$ \textbf{сходится} к $g$ \textbf{равномерно} на $X$ при $y\to y_0$, если
        $$
        \sup\limits_{x\in X}|f(x;y)-g(x)|\underset{y\to y_0}\longrightarrow0
        $$
        Записывается привычным образом $f(x;y)\underset{y\to y_0}\toto g(x)$
    \end{definition}
    \begin{corollary}[Предельный переход по параметру при условии равномерной сходимости]
        \label{Предельный переход по параметру при условии равномерной сходимости}
        Пусть $(X;\mathbb A;\mu)$~--- пространство с мерой, $\mu$ конечна.\\
        $\tilde Y$~--- метрическое пространство, $Y\subset Y$, $y_0$~--- предельная точка $Y$.\\
        Пусть $f\colon X\times Y\to\mathbb R$, $f(x;y)\underset{y\to y_0}\toto g(x)$ на $X$ и пусть
        $$\forall y\in Y~f(\bullet;y)\in L(X;\mu)$$
        Тогда $g\in L(X;\mu)$
        $$
        \lim\limits_{y\to y_0}\int_Xf(x;y)~\mathrm d\mu(x)=\int_X\lim\limits_{y\to y_0}f(x;y)~\mathrm d\mu(x)
        $$
    \end{corollary}
    \begin{proof}
        Возьмём последовательность $y_n\in Y\setminus\{y_0\}$, $y_n\to y_0$. Введём $f_n(x)=f(x;y_n)$. Тогда $f_n$ равномерно стремится к $g$ на $X$.\\
        Возьмём $\eps=1$ и получим $N$ такое что $\forall n>N~\forall x\in X~|f_n(x)-g(x)|<1$.\\
        Отсюда $|g(x)|<|f_N(x)|+1$, обе части $\in L(X;\mu)$, значит $g\in L(X;\mu)$. Тогда
        $$
        |f_n(x)|\leqslant 1+|g(x)|
        $$
        Если обозначит правую часть за $\Phi$, можно будет применить теорему Лебега.
    \end{proof}
    \begin{example}
        Условие конечности меры существенно. На множестве бесконечной меры равномерная сходимость не работает:\\
        $X=[0;+\infty)$, $f_n=\frac1n\chi_{[0;n]}$. Тогда интеграл каждой $f_n$ равен 1, что не стремится к нулю.
    \end{example}
    \begin{corollary}[Непрерывность интеграла по параметру в точке]
        \label{Непрерывность интеграла по параметру в точке}
        Пусть $(X;\mathbb A;\mu)$~--- пространство с мерой.\\
        Пусть $Y$~--- метрическое пространство, $y_0\in Y$.\\
        Пусть $f\colon X\times Y\to\mathbb R$ и пусть
        $$\forall y\in Y~f(\bullet;y)\in L(X;\mu)$$
        Пусть для почти всех $x\in X$ $f(x;\bullet)$ непрерывна в $y_0$ и пусть $f$ удовлетворяет локальному условию Лебега нв $y_0$. Тогда $\int_Xf(x;y)~\mathrm d\mu(x)$ непрерывно в $y_0$.
    \end{corollary}
    \begin{proof}
        Если $y_0$~--- изолированная точка $Y$, ничего доказывать не надо, иначе она предельная. Возьмём $g(x)=f(x;y_0)$. Всё.
    \end{proof}
    \begin{corollary}
        Если условие следствия \ref{Непрерывность интеграла по параметру в точке} верно для любой точки $y_0\in Y$, то
        $$
        \int_Xf(x;y)~\mathrm d\mu(x)\in C(Y)
        $$
    \end{corollary}
    \begin{remark}
        Полезное напоминание: если $f\in C(X\times Y)$, то $\forall x\in X~f(x;\bullet)\in C(Y)$ и $\forall y\in Y~f(\bullet;y)\in C(X)$.
    \end{remark}
    \begin{theorem}[Непрерывность интеграла по параметру на множестве]
        \label{Непрерывность интеграла по параметру на множестве}
        Пусть $X,Y$~--- метрические пространства, $X$ компактно, $\mu$~--- конечная борелевская мера на $X$, $f\in C(X\times Y)$. Тогда
        $$
        \int_Xf(x;y)~\mathrm d\mu(x)\in C(Y)
        $$
    \end{theorem}
    \begin{proof}
        Из комментария выше
        $$
        \forall y\in Y~f(\bullet;y)\in C(X)
        $$
        Также $X$~--- компакт, следовательно $f(\bullet;y)$ ограничена на $X$. $\mu$~--- борелевская, значит $f(\bullet;y)$ измерима. $\mu X<+\infty$, а значит $f(\bullet;x)\in L(X;\mu)$. Отлично, теперь интеграл $\int_Xf(x;y)~\mathrm d\mu(x)$ корректно определён.\\
        Ну что ж, осталось проверить локальное условие Лебега в каждой точке $y_0\in Y$. Давайте докажем, что
        $$
        \exists V_{y_0}~f\text{ ограничена на }X\times V_{y_0}
        $$
        (мажоранта будет константой).\\
        Ну, давайте докажем от противного. Тогда в частности $\forall n\in\mathbb N~\exists x_n\in X,y_n\in B(y_0;\frac1n)~|f(x_n;y_n)|>n$. $y_n$ стремится к $y_0$, а из $x_n$ можно выделить сходящуюся (к $x_0$) подпоследовательность $x_{n_k}$ (в силу секвенциальной компактности). Но подождите. $|f(x_{n_k};y_{n_k})|>n_k\to\infty$. А левая часть стремится к $|f(x_0;y_0)|$.
    \end{proof}
    \begin{remark}
        В частности, теорема верна для меры Лебега.
    \end{remark}
    \begin{corollary}
        Если $[a;b],\langle c;d\rangle\subset\mathbb R$, $f\in C([a;b]\times\langle c;d\rangle)$, то $\int_Xf(x;y)~\mathrm dx\in C\langle c;d\rangle$.
    \end{corollary}
    \begin{example}
        Локальное условие Лебега в следствии \ref{Непрерывность интеграла по параметру в точке} и компактность $X$ в теореме \ref{Непрерывность интеграла по параметру на множестве} существенны.\\
        Пусть $X=Y=\mathbb R$,
        $$
        f(x;y)=\begin{cases}
            0 & y=0\\
            \frac1{1+\left(x+\frac1{|y|}\right)^2} & y\neq0
        \end{cases}
        $$
        Верно ли, что $f\in C(\mathbb R^2)$? Если $y_0\neq0$, то в точке $(x_0;y_0)$ всё понятно. Что с $y=0$? Ну, рассмотрим $(x_0;0)$ в окрестности $|xy|<\frac12$. Тогда
        $$
        0\leqslant f(x;y)\leqslant\frac{y^2}{y^2+(x|y|+1)^2}\leqslant4y^2\underset{(x;y)\to(x_0;0)}\longrightarrow0
        $$
        Обозначим
        $$
        I(y)=\int_{\mathbb R} f(x;y)~\mathrm dx
        $$
        Очевидно, $I(0)=0$. А если $y\neq0$, то
        $$
        I(y)=\int_{\mathbb R}\frac1{1+\left(x+\frac1{|y|}\right)^2}~\mathrm dx
        $$
        От $y$ эта штука не зависит никак, потому что сдвиг. А значит $y$ можно выкинуть, и получить, что $I(y)=\pi$. Ой. Разрыв в нуле.
    \end{example}
    \begin{theorem}[Дифференцируемость интеграла по параметру]
        \label{Дифференцируемость интеграла по параметру}
        Пусть $Y=\langle c;d\rangle\subset\mathbb R$, $(X;\mathbb A;\mu)$~--- пространство с мерой, $f\colon X\times Y\to\mathbb R$, $\forall y\in Y~f(\bullet;y)\in L(X;\mu)$, при почти всех $x\in X~f(x;\bullet)$ дифференцируемо на $Y$. Пусть $y_0\in Y$ и $\frac{\partial f}{\partial y}$ удовлетворяет локальному условию Лебега. Тогда
        $$
        \exists\left(\int_X f(x;y)~\mathrm d\mu(x)\right)'_y\bigg|_{y=y_0}=\int_Xf'_y(x;y_0)~\mathrm d\mu(x)
        $$
    \end{theorem}
    \begin{proof}
        Производную будем искать по определению. Возьмём $h\neq0$, $y_0+h\in Y$. Рассмотрим
        $$
        F(x;h)=\frac{f(x;y_0+h)-f(x;y_0)}h
        $$
        Из дифференцируемости $f(x;\bullet)$ почти везде, при почти всех $x$ $F(x;h)\underset{h\to0}\longrightarrow f'_y(x;y_0)$. Пусть
        $$
        I(y)=\int_Xf(x;y)\mathrm d\mu(x)
        $$
        Тогда
        $$
        \frac{I(y_0+h)-I(y_0)}h=\int_XF(x;h)~\mathrm d\mu(x)
        $$
        Очень хочется сделать переход под знаком интеграла. Чтобы так было можно сделать по теореме \ref{Предельный переход по параметру под знаком интеграла}, надо проверить локальное условие Лебега для $F$ в точке $h$. По теореме Лагранжа
        $$
        \exists\theta\in(0;1)~F(x;h)=f'_y(x;y+\theta h)
        $$
        Какое-то условие Лебега нам дано ($f'_y\in L_{\mathrm{loc}}$ в $y_0$). Запишем его подробно:
        $$
        \exists\Phi\in L(X;\mu)~\exists V_{y_0}~\text{для почти всех }x\in X~\forall y\in{\dot V}_{y_0}\cap Y~|f'_y(x;y)|\leqslant\Phi(x)
        $$
        Понятно, что
        $$
        \exists\delta>0~\forall h\in(-\delta;\delta)\setminus\{0\}~y_0+\theta h\in{\dot V}_{y_0}\cap Y
        $$
        Тогда
        $$
        |F(x;h)|=|f'(x;y_0+\theta h)|\leqslant\Phi(x)
        $$
    \end{proof}
    \begin{corollary}
        Если $X$~--- компакт, $\mu$~--- конечная борелевская мера на $X$, $Y=\langle c;d\rangle\subset\mathbb R$, $f,f'_y\in C(X\times Y)$. Тогда
        $$
        \int_X f(x;y)~\mathrm d\mu(x)\in C^{(1)}(Y)
        $$
        и для любой $y_0\in Y$ верно правило Лейбница.
    \end{corollary}
    \begin{proof}
        Надо проверить, локальное условие Лебега для производной. Для любой $y_0$ возьмём 
        $$
        \delta>0~[y_0-\delta;y_0+\delta]\cap\langle c;d\rangle=[\alpha;\beta]\text{ ограничено}
        $$
        Тогда $f'_y\in C(X\times[\alpha;\beta])$, а значит $f'_y$ ограничена на $X\times[\alpha;\beta]$ (мажоранта~--- постоянная). По теореме \ref{Дифференцируемость интеграла по параметру} $I'(y_0)$ равно тому, чему хочется, а \ref{Непрерывность интеграла по параметру на множестве}.
    \end{proof}
    \begin{example}
        Условие Лебега в теореме \ref{Дифференцируемость интеграла по параметру} и компактность в следствии 1 существенны.\\
        Пусть $X=(0;1]$, $Y=[0;+\infty)$, $f(x;y)=\ln(x^2+y^2)$. Тогда
        \[\begin{split}
            I(y)&=\int_0^1\ln(x^2+y^2)~\mathrm dx\overset{y\neq0}=x\ln(x^2+y^2)\Big|_{x=0}^1-\int_0^1\frac{2x^2}{x^2+y^2}~\mathrm dx=\\
            &=\ln(1+y^2)-2+2y^2\frac1y\atan\frac xy\Big|_{x=0}^1=\ln(1+y^2)-2+2y\atan\frac1y
        \end{split}\]
        А при $y=0$ это просто $-2$.\\
        Теперь давайте считать производную в нуле (где нарушено правило Лейбница)
        $$
        I'(0)=I'_+(0)=0-0+\pi
        $$
        Но
        $$
        \int_0^1\left(\ln(x^2+y^2)\right)'_y\bigg|_{y=0}~\mathrm dx=\int_0^10~\mathrm dx=0
        $$
    \end{example}
    \subparagraph{Интеграл комплекснозначной функции.}
    \begin{definition}
        Пусть $(X;\mathbb A;\mu)$~--- пространство с мерой, $E\in\mathbb A$, $f\colon E\to\mathbb C$ (или даже $\overline{\mathbb C}$). Пусть $f=u+\im v$.\\
        $f$ называется \textbf{измеримой} на $E$, если $u$ и $v$ измеримы на $E$.\\
        $f$ называется \textbf{измеримой} на $E$, если $u$ и $v$ измеримы на $E$.\\
        Положим
        $$
        \int_Ef~\mathrm d\mu=\int_Eu~\mathrm d\mu+\im\int_Ev~\mathrm d\mu
        $$
        если правая часть имеет смысл.
    \end{definition}
    \begin{property}
        Очевидно.
        $$
        \int_E\overline f~\mathrm d\mu=\overline{\int_E\overline f~\mathrm d\mu}
        $$
    \end{property}
    \begin{property}
        Арифметические свойства интеграла переносятся очевидно.
    \end{property}
    \begin{lemma}
        Пусть $f=u+\im v$, $f\in S(E)$. Тогда
        \begin{enumerate}
            \item $|f|\in S(E)$.
            \item Суммируемость $f$ и $|f|$ равносильны.
            \item
            $$
            \left|\int_Ef~\mathrm d\mu\right|\leqslant\int_E|f|~\mathrm d\mu
            $$
        \end{enumerate}
    \end{lemma}
    \begin{proof}
        \begin{enumerate}
            \item $|f|=\sqrt{u^2+v^2}$. А правая часть $\in S(E)$ по арифметическим действиям с измеримыми функциями.
            \item Из
            $$
            |u|,|v|\leqslant |f|\leqslant |u|+|v|
            $$
            Отсюда всё понятно.
            \item Если $\int_Ef~\mathrm d\mu=0$ или $\infty$, то всё понятно. Иначе
            $$
            \int_Ef~\mathrm d\mu\in \mathbb C\setminus\{0\}
            $$
            Тогда пусть
            $$
            z=\frac{\left|\int_Ef~\mathrm d\mu\right|}{\int_Ef~\mathrm d\mu}
            $$
            Понятно, что $|z|=1$. Тогда
            $$
            \left|\int_Ef~\mathrm d\mu\right|=z\int_Ef~\mathrm d\mu=\int_E zf~\mathrm d\mu
            $$
            При этом левая штука $\in\mathbb R$, а значит правая~--- тоже. Отсюда
            $$
            \left|\int_Ef~\mathrm d\mu\right|=\Re\int_E zf~\mathrm d\mu=\int_E \Re zf~\mathrm d\mu\leqslant\int_E|zf|~\mathrm d\mu=\int_E|f|~\mathrm d\mu
            $$
        \end{enumerate}
    \end{proof}
    \begin{property}
        Из леммы переносятся теоремы \hyperref[Теорема Фубини]{Фубини}, Лебега о мажорируемой сходимости и все теоремы этого параграфа. При дифференцируемость сейчас разберёмся.
    \end{property}
    \begin{theorem}[Голоморфность интеграла по параметру]
        \label{Голоморфность интеграла по параметру}
        Пусть $(X;\mathbb A;\mu)$~--- пространство с мерой, $Y\subset\mathbb C$, $f\colon X\times Y\to\mathbb C$,
        $$
        \forall y\in Y~f(\bullet;y)\in L(X;\mu)
        $$
        $$
        \text{При почти всех }\forall x\in X~f(x;\bullet)\in \scriptA(Y)
        $$
        И пусть ещё
        $$
        \forall y_0\in Y~f'_y\in L_{\mathrm{loc}}\text{ в точке }y_0
        $$
        тогда $I\in\scriptA(Y)$ и верно правило Лейбница.
    \end{theorem}
    \begin{proof}
        Единственное отличие доказательства от доказательства \ref{Дифференцируемость интеграла по параметру} в том, что
        $$
        |F(x;h)|\leqslant|f'_y(x;y_0+\theta h)|
        $$
        Этого нам хватит, так как нам нужна мажоранта.
    \end{proof}
    \begin{example}
        Пусть $\Gamma$ замкнуто в $\mathbb C$. Пусть $\mu$~--- борелевская мера на $\Gamma$. Пусть $G=\mathbb C\setminus\Gamma$, $h\in L(\Gamma;\mu)$.\\
        Пусть
        $$
        F(z)=\int_\Gamma\frac{h(\zeta)}{\zeta-z}~\mathrm d\mu(\zeta),\qquad z\in G
        $$
        тогда $F\in\scriptA(G)$ и
        $$
        \forall n\in\mathbb N~\forall z\in G~F^{(n)}(z)=n!\int_\Gamma\frac{h(\zeta)}{(\zeta-z)^{n+1}}~\mathrm d\mu(\zeta)
        $$
        Почему?\\
        Пусть $z_0\in\Gamma$. Пусть $2\sigma=\rho(z_0;\Gamma)>0$. Если $|z-z_0|<\sigma$, а $\zeta\in\Gamma$, то $|\zeta-z|\geqslant\sigma$. Тогда
        $$
        \left|\frac{\partial^n}{\partial z^n}\frac{h(\zeta)}{\zeta-z}\right|=n!\frac{|h(\zeta)|}{|\zeta-z|^{n+1}}\leqslant\underbrace{\frac{n!}{\sigma^{n+1}}|h(\zeta)|}_{\in L(\Gamma;\mu)}
        $$
        Значит можно дифференцировать сколько угодно раз.
    \end{example}
    \subparagraph{Примеры вычисления интегралов.}
    \begin{example}
        $$
        I_n=\int\limits_0^{+\infty}\frac{\mathrm dx}{(1+x^2)^{n+1}}\qquad n\in\mathbb Z_+
        $$
        А давайте введём вот такую штуку:
        $$
        I(y)=\int\limits_0^{+\infty}\frac{\mathrm dy}{y+x^2}
        $$
        Это мы считать умеем. А зачем? В потому что если продифференцировать это по $y$ $n$ раз, то получится то, что мы хотели (только с точностью до знака и факториала):
        $$
        \frac{\partial^n}{\partial y^n}\frac1{y+x^2}=\frac{(-1)^nn!}{(y+x^2)^{n+1}}
        $$
        Хорошо, а почему можно дифференцировать под знаком интеграла? Пусть $V_{y_0}=\left(\frac{y_0}2;+\infty\right)$. Тогда
        $$
        \forall y\in V_{y_0}~\forall x\in[0;+\infty)~\left|\frac{\partial^n}{\partial y^n}\frac1{y+x^2}\right|\leqslant\frac{n!}{\left(\frac{y_0}2+x^2\right)^{n+1}}=\Phi_{y_0}(x)
        $$
        Где $\Phi_{y_0}\in L[0;+\infty)$. Ну, о'кей. Давайте считать $I(y)$.
        $$
        I(y)=\frac1{\sqrt y}\atan\frac x{\sqrt y}\Big|_{x=0}^{+\infty}=\frac\pi{2\sqrt y}
        $$
        Тогда
        $$
        I^{(n)}(y)=\frac\pi2\frac{(-1)^n\frac12\cdot\frac32\cdot\cdots\cdot\frac{2n-1}2}{y^{n+1/2}}=\frac{(-1)^n(2n-1)!!}{2^{n+1}y^{n+1/2}}\pi
        $$
        Тогда
        $$
        I_n=\frac{(-1)^n}{n!}I^{(n)}(1)=\frac{(2n-1)!!}{(2n)!!}\frac\pi2
        $$
    \end{example}
    \begin{example}
        Давайте возьмём интеграл из предыдущего примера, и начнём делать с ним тёмную магию
        $$
        \int\limits_0^{+\infty}\frac{\mathrm dx}{(1+x^2)^{n+1}}=\frac{(2n-1)!!}{(2n)!!}\frac\pi2
        $$
        Давайте сделаем замену $x=\frac t{\sqrt{n+1}}$. Тогда
        $$
        \int\limits_0^{+\infty}\frac{\mathrm dx}{(1+\frac{t^2}{n+1})^{n+1}}=\frac{(2n-1)!!}{(2n)!!}\frac\pi2\sqrt{n+1}
        $$
        И теперь давайте устремим $n$ у бесконечности. В правой части по формуле Валлиса $\frac{\sqrt\pi}2$. А в левой части получится замечательный предел, и мы получим интеграл Пуассона:
        $$
        \int\limits_0^{+\infty}e^{-t^2}~\mathrm dt
        $$
        Остаётся только понять, почему можно переходить к пределу под знаком интеграла. Ну, заметим, что
        $$
        f_n(t)=\left(1+\frac{t^2}{n+1}\right)^{-n-1}
        $$
        убывает по $n$. А тогда 
        $$
        0\leqslant f_n(t)\leqslant f_0(t)=\frac1{1+t^2}
        $$
        Где $f_0\in L([0;+\infty))$.
    \end{example}
    \begin{remark}
        Поговорим о несобственных интегралах.\\
        А что о них говорить-то? А то, что у нас были несобственные интегралы в смысле Римана (это предельчик), но тут у нас интеграл по множеству, и никто не заставляет множество иметь конечную меру.\\
        Надо как-то связать несобственный интеграл Римана и интеграл по множеству.
    \end{remark}
    \begin{definition}
        Пусть $a\in\mathbb R$, $f\colon [a;+\infty)\to\mathbb R$, и $\forall A\in(a;+\infty)~\exists (L)\int\limits_a^Af$. Тогда положим
        $$
        \int\limits_a^{\to+\infty}f=\lim\limits_{A\to+\infty}\int\limits_a^Af
        $$
    \end{definition}
    \begin{lemma}
        Если существует $(L)\int\limits_a^{+\infty}f$, то существует $\int\limits_a^{\to+\infty} f$, равный собственному лебеговому интегралу.
    \end{lemma}
    \begin{proof}
        Достаточно доказать для $f\geqslant 0$ (иначе рассмотреть $f_+$ и $f_-$, а потом $\Re f$ и $\Im f$). В таком случае оба интеграла существуют в $[0;+\infty]$.\\
        Рассмотрим $f_n=f\cdot\chi_{[a;n]}$, где $n>a$. Тогда $f_n$ возрастают (по $n$) и стремятся к $f$. Тогда с одной стороны
        $$
        \int\limits_a^nf=(L)=\int\limits_a^{+\infty}f_n\underset{n\to\infty}{\overset{\text{Леви}}\longrightarrow}(L)\int\limits_a^{+\infty}f
        $$
        С другой стороны по определению несобственного интеграла левая часть стремится к несобственному интегралу.
    \end{proof}
    \begin{claim}
        $\int\limits_a^{\to+\infty}f$ сходится абсолютно тогда и только тогда, когда $f\in L[a;+\infty)$.
    \end{claim}
    \begin{proof}
        Если $f$ суммируема, то и модуль тоже. Поэтому несобственный интеграл сходится абсолютно. Аналогично обратное.
    \end{proof}
    \begin{example}
        \begin{itemize}
            \item $\int\limits_1^{+\infty}\frac{\mathrm dx}{x^2}=1$ можно рассматривать как Лебегов интеграл или как сходящийся несобственный.
            \item $\int\limits_1^{+\infty}\frac{\mathrm dx}{x}=+\infty$ можно рассматривать как Лебегов интеграл или как расходящийся несобственный.
            \item $\int\limits_0^{+\infty}\frac{\sin x}{x}~\mathrm dx=\frac\pi2$ можно рассматривать как сходящийся несобственный, но не как Лебегов.
        \end{itemize}
    \end{example}
    \begin{remark}
        Аналогично определяются несобственные интегралы для других типов промежутков или даже для многомерных интегралов. Как, например, можно трактовать такое?
        $$
        \iint\limits_{\to\mathbb R^2}f
        $$
        Ну, рассматривать $\{G_k\}$ открытые, $G_k\subset G_{k+1}$, и объединение их всех~--- $\mathbb R^2$.\\
        Так вот это не даст нам ничего нового. Почему? Докажем, что условно сходящихся интегралов не бывает. Почему? Ну, потому что если такой бывает, то у нас разошлись интегралы $f_+$ и $f_-$. Тогда мы можем взять места, где $f$ положительно, взять их столько, чтобы в интеграле получилось $>1$ и соединить эти области перемычками. Потом сделаем то же с отрицательными частями так, чтобы они в сумме с положительными давали $<-2$. И так далее. Получим, что предела $f$ нет.\\
        В итоге рассматривают только какие-то специфичные $G_n$. Типа предел по квадратам или по кому-нибудь ещё.
    \end{remark}
    \begin{remark}
        Есть много замечательных теорем о хороших свойствах несобственных интегралов, их можно прочитать в Фихтенгольце или где-нибудь ещё, а мы расскажем только одну лемму.
    \end{remark}
    \begin{lemma}
        Пусть $a\in\mathbb R$, $f\in C[a;+\infty)$, $\int\limits_a^{+\infty}f$ сходится. Введём
        $$
        I(y)=\int\limits_a^{\to+\infty}e^{-yx}f(x)~\mathrm dx
        $$
        Тогда $I\in C[0;+\infty)$.
    \end{lemma}
    \begin{proof}
        Докажем сначала, что интеграл сходится. Попутно оценим остаток. Рассмотрим $A>a$ и проинтегрируем по частям:
        $$
        \int\limits_A^{\to+\infty}e^{-yx}f(x)~\mathrm dx=\underbrace{e^{-yx}(F(x)-F(A))\bigg|_{X=A}^{+\infty}}_0+\int\limits_a^{\to+\infty}y\underbrace{e^{-yx}}_{\text{интеграл сходится}}\underbrace{(F(x)-F(A))}_{\text{ограничена}}~\mathrm dx
        $$
        Теперь докажем непрерывность интеграла в точке $y_0\geqslant0$. Рассмотрим $\eps>0$ и подберём
        $$
        A>a~\left|\int\limits_A^{+\infty}f\right|<\frac\eps3
        $$
        Тогда
        $$
        \left|\int\limits_A^{\to+\infty}e^{-yx}f(x)~\mathrm dx\right|=\left|\int\limits_a^{\to+\infty}ye^{-yx}\underbrace{(F(x)-F(A))}_{\leqslant\frac\eps3}~\mathrm dx\right|\leqslant\frac\eps3e^{-Ay}\leqslant\frac\eps3
        $$
        То есть $I_A(y)=\int\limits_a^Ae^{-yx}f(x)~\mathrm dx$ непрерывна. Остаётся рассмотреть такое $\delta>0$, что $\forall y\geqslant0:|y-y_0|<\delta~|I_A(y)-I_A(y_0)|<\frac\eps3$, тогда
        $$
        |I(y)-I(y_0)|\leqslant|I(y)-I_A(y)|+|I_A(y)-I_A(y_0)|+|I_A(y_0)-I(y_0)|<\eps
        $$
    \end{proof}
\end{document}