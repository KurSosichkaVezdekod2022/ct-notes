\documentclass{article}
\usepackage{ifluatex}
\ifluatex 
    \usepackage{fontspec}
    \setsansfont{CMU Sans Serif}%{Arial}
    \setmainfont{CMU Serif}%{Times New Roman}
    \setmonofont{CMU Typewriter Text}%{Consolas}
    \defaultfontfeatures{Ligatures={TeX}}
\else
    \usepackage[T2A]{fontenc}
    \usepackage[utf8]{inputenc}
\fi
\usepackage[english,russian]{babel}
\usepackage{amssymb,latexsym,amsmath,amscd,mathtools,wasysym}
\usepackage[shortlabels]{enumitem}
\usepackage[makeroom]{cancel}
\usepackage{graphicx}
\usepackage{geometry}
\usepackage{verbatim}
\usepackage{fvextra}

\usepackage{longtable}
\usepackage{multirow}
\usepackage{multicol}
\usepackage{tabu}
\usepackage{arydshln} % \hdashline and :

\usepackage{float}
\makeatletter
\g@addto@macro\@floatboxreset\centering
\makeatother
\usepackage{caption}
\usepackage{csquotes}
\usepackage[bb=dsserif]{mathalpha}
\usepackage[normalem]{ulem}

\usepackage[e]{esvect}
\let\vec\vv

\usepackage{xcolor}
\colorlet{darkgreen}{black!25!blue!50!green}


%% Here f*cking with mathabx
\DeclareFontFamily{U}{matha}{\hyphenchar\font45}
\DeclareFontShape{U}{matha}{m}{n}{
    <5> <6> <7> <8> <9> <10> gen * matha
    <10.95> matha10 <12> <14.4> <17.28> <20.74> <24.88> matha12
}{}
\DeclareSymbolFont{matha}{U}{matha}{m}{n}
\DeclareFontFamily{U}{mathb}{\hyphenchar\font45}
\DeclareFontShape{U}{mathb}{m}{n}{
    <5> <6> <7> <8> <9> <10> gen * mathb
    <10.95> matha10 <12> <14.4> <17.28> <20.74> <24.88> mathb12
}{}
\DeclareSymbolFont{mathb}{U}{mathb}{m}{n}

\DeclareMathSymbol{\defeq}{\mathrel}{mathb}{"15}
\DeclareMathSymbol{\eqdef}{\mathrel}{mathb}{"16}


\usepackage{trimclip}
\DeclareMathOperator{\updownarrows}{\clipbox{0pt 0pt 4.175pt 0pt}{$\upuparrows$}\hspace{-.825px}\clipbox{0pt 0pt 4.175pt 0pt}{$\downdownarrows$}}
\DeclareMathOperator{\downuparrows}{\clipbox{0pt 0pt 4.175pt 0pt}{$\downdownarrows$}\hspace{-.825px}\clipbox{0pt 0pt 4.175pt 0pt}{$\upuparrows$}}

\makeatletter
\providecommand*\deletecounter[1]{%
    \expandafter\let\csname c@#1\endcsname\@undefined}
\makeatother


\usepackage{hyperref}
\hypersetup{
    %hidelinks,
    colorlinks=true,
    linkcolor=darkgreen,
    urlcolor=blue,
    breaklinks=true,
}

\usepackage{pgf}
\usepackage{pgfplots}
\pgfplotsset{compat=newest}
\usepackage{tikz,tikz-3dplot}
\usepackage{tkz-euclide}
\usetikzlibrary{calc,automata,patterns,angles,quotes,backgrounds,shapes.geometric,trees,positioning,decorations.pathreplacing}
\pgfkeys{/pgf/plot/gnuplot call={T: && cd TeX && gnuplot}}
\usepgfplotslibrary{fillbetween,polar}
\ifluatex
\usetikzlibrary{graphs,graphs.standard,graphdrawing,quotes,babel}
\usegdlibrary{layered,trees,circular,force}
\else
\errmessage{Run with LuaTeX, if you want to use gdlibraries}
\fi
\makeatletter
\newcommand\currentnode{\the\tikz@lastxsaved,\the\tikz@lastysaved}
\makeatother

%\usepgfplotslibrary{external} 
%\tikzexternalize

\makeatletter
\newcommand*\circled[2][1.0]{\tikz[baseline=(char.base)]{
        \node[shape=circle, draw, inner sep=2pt,
        minimum height={\f@size*#1},] (char) {#2};}}
\makeatother

\newcommand{\existence}{{\circled{$\exists$}}}
\newcommand{\uniqueness}{{\circled{$\hspace{0.5px}!$}}}
\newcommand{\rightimp}{{\circled{$\Rightarrow$}}}
\newcommand{\leftimp}{{\circled{$\Leftarrow$}}}

\DeclareMathOperator{\sign}{sign}
\DeclareMathOperator{\Cl}{Cl}
\DeclareMathOperator{\proj}{pr}
\DeclareMathOperator{\Arg}{Arg}
\DeclareMathOperator{\supp}{supp}
\DeclareMathOperator{\diag}{diag}
\DeclareMathOperator{\tr}{tr}
\DeclareMathOperator{\rank}{rank}
\DeclareMathOperator{\Lat}{Lat}
\DeclareMathOperator{\Lin}{Lin}
\DeclareMathOperator{\Ln}{Ln}
\DeclareMathOperator{\Orbit}{Orbit}
\DeclareMathOperator{\St}{St}
\DeclareMathOperator{\Seq}{Seq}
\DeclareMathOperator{\PSet}{PSet}
\DeclareMathOperator{\MSet}{MSet}
\DeclareMathOperator{\Cyc}{Cyc}
\DeclareMathOperator{\Hom}{Hom}
\DeclareMathOperator{\End}{End}
\DeclareMathOperator{\Aut}{Aut}
\DeclareMathOperator{\Ker}{Ker}
\DeclareMathOperator{\Def}{def}
\DeclareMathOperator{\Alt}{Alt}
\DeclareMathOperator{\Sim}{Sim}
\DeclareMathOperator{\Int}{Int}
\DeclareMathOperator{\grad}{grad}
\DeclareMathOperator{\sech}{sech}
\DeclareMathOperator{\csch}{csch}
\DeclareMathOperator{\asin}{\sin^{-1}}
\DeclareMathOperator{\acos}{\cos^{-1}}
\DeclareMathOperator{\atan}{\tan^{-1}}
\DeclareMathOperator{\acot}{\cot^{-1}}
\DeclareMathOperator{\asec}{\sec^{-1}}
\DeclareMathOperator{\acsc}{\csc^{-1}}
\DeclareMathOperator{\asinh}{\sinh^{-1}}
\DeclareMathOperator{\acosh}{\cosh^{-1}}
\DeclareMathOperator{\atanh}{\tanh^{-1}}
\DeclareMathOperator{\acoth}{\coth^{-1}}
\DeclareMathOperator{\asech}{\sech^{-1}}
\DeclareMathOperator{\acsch}{\csch^{-1}}

\newcommand*{\scriptA}{{\mathcal{A}}}
\newcommand*{\scriptB}{{\mathcal{B}}}
\newcommand*{\scriptC}{{\mathcal{C}}}
\newcommand*{\scriptD}{{\mathcal{D}}}
\newcommand*{\scriptF}{{\mathcal{F}}}
\newcommand*{\scriptH}{{\mathcal{H}}}
\newcommand*{\scriptK}{{\mathcal{K}}}
\newcommand*{\scriptL}{{\mathcal{L}}}
\newcommand*{\scriptM}{{\mathcal{M}}}
\newcommand*{\scriptP}{{\mathcal{P}}}
\newcommand*{\scriptQ}{{\mathcal{Q}}}
\newcommand*{\scriptR}{{\mathcal{R}}}
\newcommand*{\scriptT}{{\mathcal{T}}}
\newcommand*{\scriptU}{{\mathcal{U}}}
\newcommand*{\scriptX}{{\mathcal{X}}}
\newcommand*{\Cnk}[2]{\left(\begin{matrix}#1\\#2\end{matrix}\right)}
\newcommand*{\im}{{\mathbf i}}
\newcommand*{\id}{{\mathrm{id}}}
\newcommand*{\compl}{^\complement}
\newcommand*{\dotprod}[2]{{\left\langle{#1},{#2}\right\rangle}}
\newcommand\matr[1]{\left(\begin{matrix}#1\end{matrix}\right)}
\newcommand\matrd[1]{\left|\begin{matrix}#1\end{matrix}\right|}
\newcommand\arr[2]{\left(\begin{array}{#1}#2\end{array}\right)}

\DeclareMathOperator{\divby}{\scalebox{1}[.65]{\vdots}}
\DeclareMathOperator{\toto}{\rightrightarrows}
\DeclareMathOperator{\ntoto}{\not\rightrightarrows}

\newcommand{\undercolorblack}[2]{{\color{#1}\underline{\color{black}#2}}}
\newcommand{\undercolor}[2]{{\colorlet{tmp}{.}\color{#1}\underline{\color{tmp}#2}}}

\usepackage{adjustbox}

\geometry{margin=1in}
\usepackage{fancyhdr}
\pagestyle{fancy}
\fancyfoot[L]{}
\fancyfoot[C]{Иванов Тимофей}
\fancyfoot[R]{\pagename\ \thepage}
\fancyhead[L]{}
\fancyhead[R]{\leftmark}
\renewcommand{\sectionmark}[1]{\markboth{#1}{}}

\setcounter{tocdepth}{5}
\usepackage{amsthm}
\usepackage{chngcntr}

\theoremstyle{definition}
\newtheorem{definition}{Определение}
\counterwithin*{definition}{section}

\theoremstyle{plain}
\newtheorem{theorem}{Теорема}
\counterwithin*{theorem}{section} % Without changing appearance
\newtheorem{lemma}{Лемма}
\counterwithin*{lemma}{section}
\newtheorem{corollary}{Следствие}[theorem]
\counterwithin{corollary}{theorem} % Changing appearance
\counterwithin{corollary}{lemma}
\newtheorem*{claim}{Утверждение}
\newtheorem{property}{Свойство}[definition]

\theoremstyle{remark}
\newtheorem*{remark}{Замечание}
\newtheorem*{example}{Пример}


%\renewcommand\qedsymbol{$\blacksquare$}

\counterwithin{equation}{section}


\fancyhead[L]{Математический анализ}

\let\eps\varepsilon

\undef\limsup
\DeclareMathOperator*{\limsup}{\overline{\lim}}
\undef\liminf
\DeclareMathOperator*{\liminf}{\underline{\lim}}

\let\tmp\varphi
\let\varphi\phi
\let\phi\tmp
\undef\tmp

\begin{document}
    \tableofcontents
    \begin{definition}
        Пусть $E\subset \mathbb R^n$ $f\colon E\to[0;+\infty]$. \textbf{Подграфиком} $f$ называется множество
        $$
        Q_f=\{(x;y)\in\mathbb R^{n+1}\mid x\in E,0\leqslant y\leqslant f(x)\}
        $$
    \end{definition}
    \begin{definition}
        Пусть $E\subset \mathbb R^n$ $f\colon E\to\overline{\mathbb R}$. \textbf{Графиком} $f$ называется множество
        $$
        \Gamma_f=\{(x;f(x))\in\mathbb R^{n+1}\mid x\in E\}
        $$
    \end{definition}
    \begin{remark}
        Отличается от общего определения тем, что $\Gamma_f\subset\mathbb R^{n+1}$
    \end{remark}
    \begin{theorem}[О мере графика]
        \label{О мере графика}
        Пусть $E\in\mathbb A_n$, $f\in S(E)$. Тогда $\Gamma_f\in\mathbb A_{n+1}$ и $\mu_{n+1}\Gamma_f=0$.
    \end{theorem}
    \begin{proof}
        \begin{itemize}
            \item Сначала разберём случай, когда $\mu E<+\infty$. Заключим $\Gamma_f$ в множество сколь угодно малой меры. Зафиксируем $\eps>0$. Пусть
            $$
            e_k=E(k\eps<f(k+1)\eps)
            $$
            Тогда
            $$
            E=E(|f|=+\infty)\cup\bigcup\limits_{k\in\mathbb Z}e_k
            $$
            Тогда
            $$
            \Gamma_f=\bigcup\limits_{k\in\mathbb Z}\Gamma_{f\big|_{e_k}}\subset\bigcup\limits_{k\in\mathbb Z}e_k\times[k\eps;(k+1)\eps)=H_\eps
            $$
            Заметим, что
            $$
            \mu_{n+1}H_\eps=\sum\limits_{k\in\mathbb Z}\mu_ne_k\cdot\eps\leqslant\mu_nE\eps
            $$
            По критерию измеримости утверждение теоремы верно.
            \item Теперь пусть $\mu E=+\infty$. По $\sigma$-конечности $\mu_n$
            $$
            E=\bigcup\limits_{j=1}^\infty E_j\qquad\mu_nE_j<+\infty
            $$
            А значит $f\Big|_{E_j}$ имеет измеримый график нулевой меры, а поскольку
            $$
            \Gamma_f=\bigcup\limits_{j=1}^\infty \Gamma_{f\big|_{E_j}}
            $$
            Верно требуемое.
        \end{itemize}
    \end{proof}
    \begin{theorem}
        \label{О мере подграфика}
        Пусть $E\in\mathbb A_n$, $f\colon E\to[0;+\infty]$. Тогда измеримость $f$ и её подграфика равносильны и в случае измеримости имеет место равенство
        $$
        \mu_{n+1}Q_f=\int_Ef~\mathrm d\mu_n
        $$
    \end{theorem}
    \begin{proof}
        Пусть нам известна измеримость подграфика. Тогда искомая формула следует из принципа Кавальери:
        $$
        Q_f(x)=\begin{cases}
            \varnothing & x\notin E\\
            [0;f(x)\rangle & x\in E
        \end{cases}
        $$
        Отсюда
        $$
        \mu_1Q_f(x)=\begin{cases}
            0 & x\notin E\\
            f(x) & x\in E
        \end{cases}
        $$
        Отсюда если $Q_f$ измеримо, то формула следует из принципа Кавальери. А также в принципе Кавальери в качестве следствия был факт, что функция $x\mapsto\mu_1Q_f(x)$ измерима, а значит и $f$ измерима как сужение $x\mapsto\mu_1Q_f(x)$ на $E$.\\
        Осталось доказать, что если $f$ измерима, то её подграфик измерим. Рассмотрим случаи:
        \begin{enumerate}
            \item $f$ простая.
            $$
            f=\sum\limits_{k=1}^Nc_k\chi_{A_k}\qquad A_k\in\mathbb A_n,c_k\in[0;+\infty)
            $$
            Можно считать, что $A_k$ дизъюнктны. И ещё можно считать, что $A_k\subset E$ и в объединении дают $E$. Тогда
            $$
            Q_f=\bigsqcup_{k=1}^NA_k\times[0;c_k]
            $$
            Отсюда следует измеримость.
            \item Общий случай: $f$ произвольная неотрицательная измеримая функция. Приблизим её возрастающей последовательность простых функций $\phi_n$. Проверим, что 
            $$
            Q_f=\bigcup Q_{\phi_n}\cup\Gamma_f
            $$
            Тогда мы докажем искомое.
            \begin{itemize}
                \item[$\supset$] ясно т.к. $\phi_n\leqslant f\Rightarrow Q_{\phi_n}\subset Q_f$.
                \item[$\subset$] рассмотрим $(x;y)\in Q_f$. То есть $x\in E$, $y\in[0;f(x)]$. Если $y=f(x)$, то понятно. Иначе
                $$
                \exists N\in\mathbb N~y<\phi_N(x)\Rightarrow \exists N~(x;y)\in Q_{\phi_N}
                $$
            \end{itemize}
        \end{enumerate}
    \end{proof}
    \begin{remark}
        Условие измеримости $E$ существенно. Если $f\equiv0$ не неизмеримом множестве, например, то $Q_f\in\mathbb A_{n+1}$ и $\mu_{n+1}Q_f=0$.
    \end{remark}
    \begin{theorem}[Теорема Тонелли]
        \label{Теорема Тонелли}
        Пусть $E\subset\mathbb R^{n+m}$, $f\in S(E\to[0;+\infty])$. Тогда
        \begin{enumerate}
            \item При почти всех $x\in\mathbb R^n$ функция $f(x;\bullet)\in S(E(x))$.
            \item Пусть $I(x)=\int_{E(x)}f(x;y)~\mathrm dy$. Тогда $I(x)\in S(\mathbb R^n)$.
            \item
            $$\int_E f~\mathrm d\mu_{n+m}=\int_{\mathbb R^n}I(x)~\mathrm dx$$
        \end{enumerate}
    \end{theorem}
    \begin{proof}
        По теореме \ref{О мере подграфика}, что $Q_f\in\mathbb A_{n+m+1}$ и
        $$
        \mu_{n+m+1}Q_f=\int_E f~\mathrm d\mu_{n+m}
        $$
        Воспользуемся принципом Кавальери:
        $$
        \mu_{n+m+1}Q_f=\int_{\mathbb R^n}\mu_{m+1}Q_f(x)~\mathrm dx
        $$
        Заметим, что
        \[\begin{split}
            Q_f(x)&=\left\{(y;z)\in\mathbb R^{m+1}\mid (x;y;z)\in Q_f\right\}=\\
            &=\left\{(y;z)\in\mathbb R^{m+1}\mid (x;y)\in E,z\in[0;f(x;y)]\right\}=\\
            &=\left\{(y;z)\in\mathbb R^{m+1}\mid y\in E(x),z\in[0;f(x;y)]\right\}
        \end{split}\]
        Да это же подграфик $f(x;\bullet)$!
        \begin{enumerate}
            \item По теореме Кавальери при почти всех $x\in\mathbb R^n$ $Q_f(x)$ измеримо, а значит мы доказали первое утверждение по теореме \ref{О мере подграфика}.
            \item
            $$
            \mu_{m+1}Q_f(x)=\mu_{m+1}Q_{f(x;\bullet)}\overset{\ref{О мере подграфика}}=\int_{E(x)}f(x;y)~\mathrm dy=I(x)
            $$
            Отсюда $I(x)$ измерима по всё тому же принципу Кавальери.
            \item Приравняем два выражения для $\mu_{n+m+1}Q_f$.
        \end{enumerate}
    \end{proof}
    \begin{theorem}[Теорема Фубини]
        \label{Теорема Фубини}
        \label{Выражение кратного интеграла через повторный}
        Пусть $E\subset\mathbb R^{n+m}$, $f\in L(E)$. Тогда
        \begin{enumerate}
            \item При почти всех $x\in\mathbb R^n$ функция $f(x;\bullet)\in L(E(x))$.
            \item Пусть $I(x)=\int_{E(x)}f(x;y)~\mathrm dy$. Тогда $I(x)\in L(\mathbb R^n)$.
            \item
            $$\int_E f~\mathrm d\mu_{n+m}=\int_{\mathbb R^n}I(x)~\mathrm dx$$
        \end{enumerate}
    \end{theorem}
    \begin{proof}
        Применим \hyperref[Теорема Тонелли]{теорему Тонелли} для $f_+$ и $f_-$. Пусть $I^\pm=\int_{E(x)}f_\pm(x;y)~\mathrm dy$. По теореме Тонелли
        $$
        \int_E f_\pm~\mathrm d\mu_{n+m}=\int_{\mathbb R^n}I^\pm(x)~\mathrm dx<+\infty
        $$
        Учитывая $f_{\pm}(x;\bullet)=(f(x;\bullet))_\pm$, имеем
        $$
        I^+-I^-=I\in L(\mathbb R^n)
        $$
        При почти всех $x$ $I^\pm(x)<+\infty$, а значит при почти всех $x$ $f_\pm\in L(E(x))$ отсюда $f\in L(E(x))$.
    \end{proof}
    \begin{remark}
        В \hyperref[Теорема Тонелли]{теореме Тонелли} все условия можно ослабить:
        \begin{enumerate}
            \item Если $f\in S(E)$ (не важен знак), то при почти всех $x\in \mathbb R^n$ $f(x;\bullet)\in S(E(x))$.
            \item Если $I(x)$ существует почти во всех $x\in\mathbb R^n$, то $I\in S(\mathbb R^n)$
            \item Если существует $\int_Ef~\mathrm d\mu_{n+m}\in\overline{\mathbb R}$, то верно условие пункта 2 и
            $$
            \int_Ef~\mathrm d\mu_{n+m}\in\overline{\mathbb R}=\int_{\mathbb R^n}I(x)~\mathbb dx
            $$
        \end{enumerate}
        Доказывается всё это как в теореме Фубини.
    \end{remark}
    \begin{remark}
        $$
        \left\{\begin{aligned}
            &\forall x\in\mathbb R^n~f(x;\bullet)\in S(E(x))\\
            &\forall y\in\mathbb R^m~f(\bullet;m)\in S(E(y))
        \end{aligned}\right.\not\Rightarrow f\in S(E)
        $$
        Серпинский построил пример такого неизмеримого $E\subset\mathbb R^2$, что $E$ пересекается с любой прямой не более чем в двух точках. Мы говорить о нём не будем, т.к. он довольно сложен.
    \end{remark}
    \begin{definition}
        Пусть $X\subset\mathbb R^n$, $Y\subset\mathbb R^m$, $f\colon X\to\mathbb R$, $g\colon Y\to\mathbb R$. Тогда
        $$
        f\otimes g\colon\substack{X\times Y\to\mathbb R\\(x;y)\mapsto f(x)g(y)}
        $$
    \end{definition}
    \begin{lemma}
        Если $f\in S(X)$, $g\in S(Y)$, то $f\otimes g\in S(X\times Y)$.
    \end{lemma}
    \begin{proof}
        Пусть
        $$
        \tilde f(x;y)=f(x)\qquad\tilde g(x;y)=g(y)
        $$
        Докажем, что $\tilde f$ и $\tilde g$ измеримы, тогда $f\otimes g$ будет измеримо как произведение измеримых.\\
        $$(X\times Y)(\tilde f>a)=X(\tilde f>0)\times Y$$
        Левое измеримо по измеримости $f$, правое~--- потому что, а произведение измеримы измеримо.
    \end{proof}
    \begin{corollary}
        Пусть $X\subset\mathbb R^n$, $Y\subset\mathbb R^m$. Если
        $$
        \left[\begin{aligned}
            &f\in S(X\to[0;+\infty])\land g\in S(Y\to[0;+\infty])\\
            &f\in L(X)\land g\in L(Y)
        \end{aligned}\right.
        $$
        То
        $$
        \int_{X\times Y}f\otimes g~\mathrm d\mu_{n+m}=\int_Xf~\mathrm d\mu_n\int_Yg~\mathrm d\mu_m
        $$
    \end{corollary}
    \begin{proof}
        В первом случае нет сомнений в существовании интегралов. Пусть $E=X\times Y$. Тогда
        $$
        \int_Ef\otimes g~\mathrm d\mu_{n+m}=\int_X\left(\int_Yf(x)g(y)~\mathrm dy\right)~\mathrm dx
        $$
        так как $E(x)=\begin{cases}
            \varnothing&x\notin X\\
            Y&x\in X
        \end{cases}$.\\
        А почему то же самое верны для произвольного знака, если интегралы от них конечны? Ну, чтобы применить теорему Фубини, надо проверить суммируемость $f\otimes g$. Ну, смотрите. По доказанному
        $$
        \int_{X\times Y}|f\otimes g|~\mathrm d\mu_{n+m}=\int_X|f|~\mathrm d\mu_n\int_Y|g|~\mathrm d\mu_m
        $$
        По условию оба этих интеграла конечны, значит $|f\otimes g|$ суммируема, а суммируемость функции равносильна суммируемости её модуля.
    \end{proof}
    \begin{remark}
        Мы знаем, что в условиях теорем \hyperref[Теорема Тонелли]{Тонелли} и \hyperref[Теорема Фубини]{Фубини} верно
        $$
        \int_E f~\mathrm d\mu_{n+m}=\int_{\mathbb R^n}\left(\int_{E(x)}f(x;y)~\mathrm dy\right)~\mathrm dx
        $$
        Тривиально, то же можно записать, поменяв $x$ и $y$ ролями.
        $$
        \int_E f~\mathrm d\mu_{n+m}=\int_{\mathbb R^n}\left(\int_{E(y)}f(x;y)~\mathrm dx\right)~\mathrm dy
        $$
        А значит два повторных интеграла равны.
    \end{remark}
    \begin{example}
        Повторные интегралы могут быть не равны
        $$
        f(x;y)=\frac{x^2-y^2}{(x^2+y^2)^2}\qquad E=[-1;1]^2
        $$
        Тогда
        $$
        \int\limits_{-1}^1 \frac{x^2-y^2}{(x^2+y^2)^2}~\mathrm dy=\frac y{x^2+y^2}\bigg|_{y=-1}^1=\frac2{x^2+1}
        $$
        $$
        \int\limits_{-1}^1 \int\limits_{-1}^1 \frac{x^2-y^2}{(x^2+y^2)^2}~\mathrm dy~\mathrm dx=\int\limits_{-1}^1 \frac2{x^2+1}~\mathrm dx=4\atan x\bigg|_0^1=\pi
        $$
        Другой повторный интеграл будет равен $-\pi$, как несложно заметить.
    \end{example}
    \begin{example}
        Неверно, что если повторные интегралы равны, то двойной существует.
        $$
        g(x;y)=\frac{2xy}{(x^2+y^2)^2}\qquad E=[-1;1]^2
        $$
        Поскольку функция $g$ нечётна по каждой переменной, оба повторных интеграла равны нулю. Отсюда если двойной интеграл существует, то равен нулю.\\
        Докажем, что он не существует. Для этого докажем, что $g$ не суммируема. Попробуем проинтегрировать $|g|$:
        $$
        \int\limits_{-1}^1\int\limits_{-1}^1\frac{|2xy|}{(x^2+y^2)^2}~\mathrm dy~\mathrm dx=
        4\int\limits_0^1\int\limits_0^1\frac{|2xy|}{(x^2+y^2)^2}~\mathrm dy~\mathrm dx=
        4\int\limits_0^1\frac{-x}{x^2+y^2}\bigg|_{y=0}^1~\mathrm dx=
        4\int\limits_0^1\frac{-x}{x^2+1}+\frac1x~\mathrm dx=+\infty
        $$
        Отсюда $g$ не суммируема. А значит нулю её интеграл не равен, то есть он не существует.
    \end{example}
    \begin{definition}
        Пусть $E\subset\mathbb R^{n+m}$ \textbf{Проекцией} $E$ на первое координатное пространство называется
        $$
        \operatorname{Pr}_1E=\left\{x\in\mathbb R^n\mid E(x)\neq\varnothing\right\}
        $$
    \end{definition}
    \begin{remark}
        Проекция измеримого множества может быть быть неизмеримой (достаточно добавить к измеримому двумерному множеству неизмеримое одномерное).
    \end{remark}
    \begin{definition}
        Множество
        $$
        \operatorname{Pr}^*_1E=\left\{x\in\mathbb R^n\mid \mu_m E(x)>0\right\}
        $$
        называется \textbf{существенной проекцией} множества $E$.
    \end{definition}
    \begin{property}
        Существенная проекция измерима. (Как Лебегово множество функции $\mu_m E(\bullet)$).
    \end{property}
    \begin{property}
        При $f$ подходящем под теоремы \hyperref[Теорема Тонелли]{Тонелли} и \hyperref[Теорема Фубини]{Фубини} верно
        $$
        \int_E f~\mathrm d\mu_{n+m}=\int_{\Pr^*_1}I(x)~\mathrm dx
        $$
    \end{property}
    \begin{remark}
        Теоремы \hyperref[Теорема Тонелли]{Тонелли} и \hyperref[Теорема Фубини]{Фубини} можно применять несколько раз.
    \end{remark}
    \begin{definition}
        Пусть $(X;\mathbb A;\mu)$ и $(Y;\mathbb B;\nu)$~--- пространства с мерами. Пусть
        $$
        \mathbb A\odot\mathbb B=\{A\times B\mid A\in\mathbb A,B\in\mathbb B\}
        $$
        Тогда $\mathbb A\odot\mathbb B$ является полукольцом, а
        $$
        \pi_0\colon A\times B\to\mu A\nu B
        $$
        является мерой на $\mathbb A\oplus\mathbb B$. Тогда $\pi$~--- стандартное распространение $\pi_0$ на $\sigma$-алгебру $\mathbb C$ называется произведением мер $\mu$ и $\nu$.\\
        Обозначения:
        $$
        \mathbb C=\mathbb A\otimes\mathbb B\qquad \pi=\mu\times\nu
        $$
    \end{definition}
    \begin{remark}
        Доказывать корректность определения мы не будем.
    \end{remark}
    \begin{property}
        $$
        \mu_{n+m}=\mu_n\times\mu_m
        $$
    \end{property}
    \begin{property}
        Если $\mu$ и $\nu$ являются $\sigma$-конечными, то $\mu\times\nu$~--- тоже.
    \end{property}
    \begin{property}
        Произведение мер ассоциативно.
    \end{property}
    \begin{property}
        Все теоремы этого параграфа с доказательствами верны для полных $\sigma$-конечных мер.
    \end{property}
    \begin{theorem}[Теорема Тонелли для абстрактных пространств с мерой]
        Пусть $E\subset X\times Y$, $f\in S_{\mathbb A\otimes\mathbb B}(E\to[0;+\infty])$. Тогда
        \begin{enumerate}
            \item При почти всех $x\in x$ функция $f(x;\bullet)\in S_{\mathbb B}(E(x))$.
            \item Пусть $I(x)=\int_{E(x)}f(x;\bullet)~\mathrm d\nu$. Тогда $I(x)\in S_{\mathbb A}(X)$.
            \item
            $$\int_E f~\mathrm d(\mu\times\nu)=\int_{X}I(x)~\mathrm d\mu$$
        \end{enumerate}
    \end{theorem}
    \paragraph{Замена переменной в интеграле.}
    \begin{theorem}[Общая схема замены переменной в интеграле]
        \label{Общая схема замены переменной в интеграле}
        Пусть $(X;\mathbb A;\mu)$, $(Y;\mathbb B;\nu)$~--- пространства с мерами. Пусть $h\in S_{\mathbb A}(X\to[0;+\infty])$, $\Phi\colon X\to Y$ и
        $$
        \forall B\in\mathbb B~\Phi^{-1}(B)\in\mathbb A\qquad \forall B\in\mathbb B~\nu B=\int_{\Phi^{-1}(B)}h~\mathrm d\mu
        $$
        Пусть $f\in S_{\mathbb B}(Y)$. Тогда
        \begin{enumerate}
            \item $f\circ\Phi\in S_{\mathbb A}(X)$.
            \item
            $$
            \int_Y f~\mathrm d\nu=\int_X(f\circ\Phi)h~\mathrm d\mu
            $$
            (Трактовка стандартная: интегралы существуют или нет одновременно, если существуют, то равны.)
        \end{enumerate}
    \end{theorem}
    \begin{proof}
        \begin{enumerate}
            \item Рассмотрим Лебегово множество
            $$
            X(f\circ\Phi>a)=\{x\in X\mid f(\Phi(x))>a\}=\{x\in X\mid \Phi(x)\in Y(f>a)\}=\Phi^{-1}(Y(f>a))
            $$
            По условию $Y(f>a)\in \mathbb B$, а значит $\Phi^{-1}\in\mathbb A$.
            \item Разберём случаи
            \begin{enumerate}[a.]
                \item $f=\chi_B\mid B\in\mathbb B$. Тогда
                $$
                (\chi_B\circ\Phi)(x)=\begin{cases}
                    1 & x\in\Phi^{-1}(B)\\0 & x\notin\Phi^{-1}(B)
                \end{cases}=\chi_{\Phi^{-1}(B)}(x)
                $$
                Тогда
                $$
                \int_{Y}\chi_B~\mathrm d\nu=\nu B=\int_{\Phi^{-1}(B)}h~\mathrm d\nu=\int_{X}\underbrace{\chi_{\Phi^{-1}(B)}}_{f\circ\Phi}h~\mathrm d\nu
                $$
                \item По линейности равенство верно для простых функций.
                \item Для положительных измеримых функций рассмотрим последовательность $\phi_n$, возрастающую к $f$ и перейдём к пределу в равенстве
                $$
                \int_Y\phi_n~\mathrm d\nu=\int_X(\phi_n\circ\Phi)h~\mathrm d\mu
                $$
                по теореме Леви.
                \item Для произвольных измеримых функций рассмотрим $f_{\pm}$
                $$
                \int_Yf_\pm~\mathrm d\nu=\int_X(\phi_n\circ\Phi)_\pm h~\mathrm d\mu=\int_X(\phi_n\circ\Phi h)_\pm~\mathrm d\mu
                $$
            \end{enumerate}
        \end{enumerate}
    \end{proof}
    \begin{remark}
        В условиях теоремы \ref{Общая схема замены переменной в интеграле} суммируемость $f$ по $\nu$ равносильная суммируемости $(f\circ\Phi)h$ по $\mu$
    \end{remark}
    \begin{corollary}
        В условиях теоремы \ref{Общая схема замены переменной в интеграле} если $B\in\mathbb B$, $f\in S_{\mathbb B}(B)$, то
        $$
        \int_B f~\mathrm d\nu=\int_{\Phi^{-1}(B)}(f\circ\Phi)h~\mathrm d\mu
        $$
    \end{corollary}
    \begin{proof}
        Продолжим $f$ нулём на $Y\setminus B$.
    \end{proof}
    \begin{definition}
        В условии теоремы \ref{Общая схема замены переменной в интеграле} $\nu$ называется $h$-взвешенным $\Phi$-\textbf{образом меры} $\mu$.
    \end{definition}
    \begin{remark}
        Пусть
        $$
        \mathbb A^*=\{\Phi^{-1}(B)\mid B\in\mathbb B\}
        $$
        Нетрудно заметить, что это $\sigma$-алгебра.\\
        В условиях теоремы \ref{Общая схема замены переменной в интеграле} $\mathbb A^*\subset\mathbb A$.\\
        Пусть
        $$
        \mathbb B^*=\{B\subset Y\mid \Phi^{-1}\in\mathbb A\}
        $$
        тогда условиях теоремы \ref{Общая схема замены переменной в интеграле} $\mathbb B\subset\mathbb B^*$.
    \end{remark}
    \begin{claim}
        Если
        $$
        \nu B=\int_{\Phi^{-1}(B)}h~\mathrm d\mu
        $$
        То $\nu$~--- мера на $\mathbb B$.
    \end{claim}
    \begin{proof}
        Остаётся как несложное упражнение читателю.
    \end{proof}
    \begin{example}
        $h\equiv1$~--- невзвешенный образ меры.
        $$
        \nu B=\mu\Phi^{-1}(B)\Rightarrow \int_Yf~\mathrm d\nu=\int_Xf\circ\Phi~\mathrm d\mu
        $$
    \end{example}
    \begin{example}
        $X=Y$, $\mathbb A=\mathbb B$, $\Phi=\mathrm{id}$.
        $$
        \nu A=\int_A h~\mathrm d\mu\Rightarrow \int_X f~\mathrm d\nu=\int_Xfh~\mathrm d\mu
        $$
        Тогда пишут $\mathrm d\nu=h\mathrm d\mu$.
    \end{example}
    \begin{definition}
        Если
        $$\nu A=\int_A h~\mathrm d\mu$$
        то $h$ называется \textbf{плотностью} меры $\nu$ относительны меры $\mu$.
    \end{definition}
    \begin{property}
        Если $h=\tilde h$ $\mu$-почти везде, то $\nu=\tilde\nu$. Для $\sigma$-конечных мер верно и обратное.\\
        Без доказательства.
    \end{property}
    \begin{theorem}[Критерий плотности]
        \label{Критерий плотности}
        Пусть даны $X,\mathbb A$ и $\mu$ и $\nu$~--- меры на $\mathbb A$, $h\colon S(X\to[0;+\infty])$. Тогда следующие утверждения равносильны:
        \begin{enumerate}
            \item $h$~--- плотность $\nu$ относительны $\mu$.
            \item
            $$
            \forall A\in\mathbb A~\mu A\inf\limits_Ah\leqslant\nu A\leqslant \mu A\sup\limits_Ah
            $$
        \end{enumerate}
    \end{theorem}
    \begin{proof}
        \begin{itemize}
            \item Из первого во второе ясно из оценки интеграла.
            \item Рассмотрим
            $$
            A=A(h=0)\cup A(0<h<+\infty)\cup A(h=+\infty)
            $$
            Равенство есть для первой части:
            $$\nu A(h=0)=\int_{A(h=0)} h~\mathrm d\mu$$
            так как левое равно нулю по условию второго утверждения, а правое~--- потому что функция тождественный ноль.\\
            также очевидно равенство есть для третьей части:
            $$\nu A(h=+\infty)=\begin{cases}
                +\infty & \mu A>0\\0 & \mu A\equiv0
            \end{cases}=\int_{A(h=+\infty)} h~\mathrm d\mu$$
            Далее можно считать $0<h<+\infty$ на $A$.\\
            Рассмотрим $q\in(0;1)$. Пусть
            $$
            A_j=A(q^j\leqslant h<q^{j-1})
            $$
            Очевидно, $A_j\in\mathbb A$ и $\bigsqcup\limits_{j\in\mathbb Z}A_j=A$. Нам известно, что
            $$
            q^j\mu A_j\leqslant\nu A_j\leqslant q^{j-1}\mu A_j
            $$
            А ещё из оценки интеграла
            $$
            q^j\mu A_j\leqslant\int_{A_j}h~\mathrm d\mu\leqslant q^{j-1}\mu A_j
            $$
            Отсюда
            \[\begin{split}
                q\int_Ah~\mathrm d\mu&=q\sum\limits_j\int_{A_j}h~\mathrm d\mu\leqslant\\
                &\leqslant\sum\limits_jq^j\mu A_j\leqslant\sum\limits_j\nu A_j=\\
                &=\fbox{\nu A}\leqslant\sum\limits_j q^{j-1}\mu A_j\leqslant\\
                &\leqslant\frac1q\sum\limits_j\int_{A_j}h~\mathrm d\mu=\\
                &=\frac1q\int_Ah~\mathrm d\mu
            \end{split}\]
            Если взять начало, конец и то, что в квадратике, после чего устремить $q$ к единицу, то получим искомое.
        \end{itemize}
    \end{proof}
\end{document}