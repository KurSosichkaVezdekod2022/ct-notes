\documentclass{article}
\usepackage{ifluatex}
\ifluatex 
    \usepackage{fontspec}
    \setsansfont{CMU Sans Serif}%{Arial}
    \setmainfont{CMU Serif}%{Times New Roman}
    \setmonofont{CMU Typewriter Text}%{Consolas}
    \defaultfontfeatures{Ligatures={TeX}}
\else
    \usepackage[T2A]{fontenc}
    \usepackage[utf8]{inputenc}
\fi
\usepackage[english,russian]{babel}
\usepackage{amssymb,latexsym,amsmath,amscd,mathtools,wasysym}
\usepackage[shortlabels]{enumitem}
\usepackage[makeroom]{cancel}
\usepackage{graphicx}
\usepackage{geometry}
\usepackage{verbatim}
\usepackage{fvextra}

\usepackage{longtable}
\usepackage{multirow}
\usepackage{multicol}
\usepackage{tabu}
\usepackage{arydshln} % \hdashline and :

\usepackage{float}
\makeatletter
\g@addto@macro\@floatboxreset\centering
\makeatother
\usepackage{caption}
\usepackage{csquotes}
\usepackage[bb=dsserif]{mathalpha}
\usepackage[normalem]{ulem}

\usepackage[e]{esvect}
\let\vec\vv

\usepackage{xcolor}
\colorlet{darkgreen}{black!25!blue!50!green}


%% Here f*cking with mathabx
\DeclareFontFamily{U}{matha}{\hyphenchar\font45}
\DeclareFontShape{U}{matha}{m}{n}{
    <5> <6> <7> <8> <9> <10> gen * matha
    <10.95> matha10 <12> <14.4> <17.28> <20.74> <24.88> matha12
}{}
\DeclareSymbolFont{matha}{U}{matha}{m}{n}
\DeclareFontFamily{U}{mathb}{\hyphenchar\font45}
\DeclareFontShape{U}{mathb}{m}{n}{
    <5> <6> <7> <8> <9> <10> gen * mathb
    <10.95> matha10 <12> <14.4> <17.28> <20.74> <24.88> mathb12
}{}
\DeclareSymbolFont{mathb}{U}{mathb}{m}{n}

\DeclareMathSymbol{\defeq}{\mathrel}{mathb}{"15}
\DeclareMathSymbol{\eqdef}{\mathrel}{mathb}{"16}


\usepackage{trimclip}
\DeclareMathOperator{\updownarrows}{\clipbox{0pt 0pt 4.175pt 0pt}{$\upuparrows$}\hspace{-.825px}\clipbox{0pt 0pt 4.175pt 0pt}{$\downdownarrows$}}
\DeclareMathOperator{\downuparrows}{\clipbox{0pt 0pt 4.175pt 0pt}{$\downdownarrows$}\hspace{-.825px}\clipbox{0pt 0pt 4.175pt 0pt}{$\upuparrows$}}

\makeatletter
\providecommand*\deletecounter[1]{%
    \expandafter\let\csname c@#1\endcsname\@undefined}
\makeatother


\usepackage{hyperref}
\hypersetup{
    %hidelinks,
    colorlinks=true,
    linkcolor=darkgreen,
    urlcolor=blue,
    breaklinks=true,
}

\usepackage{pgf}
\usepackage{pgfplots}
\pgfplotsset{compat=newest}
\usepackage{tikz,tikz-3dplot}
\usepackage{tkz-euclide}
\usetikzlibrary{calc,automata,patterns,angles,quotes,backgrounds,shapes.geometric,trees,positioning,decorations.pathreplacing}
\pgfkeys{/pgf/plot/gnuplot call={T: && cd TeX && gnuplot}}
\usepgfplotslibrary{fillbetween,polar}
\ifluatex
\usetikzlibrary{graphs,graphs.standard,graphdrawing,quotes,babel}
\usegdlibrary{layered,trees,circular,force}
\else
\errmessage{Run with LuaTeX, if you want to use gdlibraries}
\fi
\makeatletter
\newcommand\currentnode{\the\tikz@lastxsaved,\the\tikz@lastysaved}
\makeatother

%\usepgfplotslibrary{external} 
%\tikzexternalize

\makeatletter
\newcommand*\circled[2][1.0]{\tikz[baseline=(char.base)]{
        \node[shape=circle, draw, inner sep=2pt,
        minimum height={\f@size*#1},] (char) {#2};}}
\makeatother

\newcommand{\existence}{{\circled{$\exists$}}}
\newcommand{\uniqueness}{{\circled{$\hspace{0.5px}!$}}}
\newcommand{\rightimp}{{\circled{$\Rightarrow$}}}
\newcommand{\leftimp}{{\circled{$\Leftarrow$}}}

\DeclareMathOperator{\sign}{sign}
\DeclareMathOperator{\Cl}{Cl}
\DeclareMathOperator{\proj}{pr}
\DeclareMathOperator{\Arg}{Arg}
\DeclareMathOperator{\supp}{supp}
\DeclareMathOperator{\diag}{diag}
\DeclareMathOperator{\tr}{tr}
\DeclareMathOperator{\rank}{rank}
\DeclareMathOperator{\Lat}{Lat}
\DeclareMathOperator{\Lin}{Lin}
\DeclareMathOperator{\Ln}{Ln}
\DeclareMathOperator{\Orbit}{Orbit}
\DeclareMathOperator{\St}{St}
\DeclareMathOperator{\Seq}{Seq}
\DeclareMathOperator{\PSet}{PSet}
\DeclareMathOperator{\MSet}{MSet}
\DeclareMathOperator{\Cyc}{Cyc}
\DeclareMathOperator{\Hom}{Hom}
\DeclareMathOperator{\End}{End}
\DeclareMathOperator{\Aut}{Aut}
\DeclareMathOperator{\Ker}{Ker}
\DeclareMathOperator{\Def}{def}
\DeclareMathOperator{\Alt}{Alt}
\DeclareMathOperator{\Sim}{Sim}
\DeclareMathOperator{\Int}{Int}
\DeclareMathOperator{\grad}{grad}
\DeclareMathOperator{\sech}{sech}
\DeclareMathOperator{\csch}{csch}
\DeclareMathOperator{\asin}{\sin^{-1}}
\DeclareMathOperator{\acos}{\cos^{-1}}
\DeclareMathOperator{\atan}{\tan^{-1}}
\DeclareMathOperator{\acot}{\cot^{-1}}
\DeclareMathOperator{\asec}{\sec^{-1}}
\DeclareMathOperator{\acsc}{\csc^{-1}}
\DeclareMathOperator{\asinh}{\sinh^{-1}}
\DeclareMathOperator{\acosh}{\cosh^{-1}}
\DeclareMathOperator{\atanh}{\tanh^{-1}}
\DeclareMathOperator{\acoth}{\coth^{-1}}
\DeclareMathOperator{\asech}{\sech^{-1}}
\DeclareMathOperator{\acsch}{\csch^{-1}}

\newcommand*{\scriptA}{{\mathcal{A}}}
\newcommand*{\scriptB}{{\mathcal{B}}}
\newcommand*{\scriptC}{{\mathcal{C}}}
\newcommand*{\scriptD}{{\mathcal{D}}}
\newcommand*{\scriptF}{{\mathcal{F}}}
\newcommand*{\scriptH}{{\mathcal{H}}}
\newcommand*{\scriptK}{{\mathcal{K}}}
\newcommand*{\scriptL}{{\mathcal{L}}}
\newcommand*{\scriptM}{{\mathcal{M}}}
\newcommand*{\scriptP}{{\mathcal{P}}}
\newcommand*{\scriptQ}{{\mathcal{Q}}}
\newcommand*{\scriptR}{{\mathcal{R}}}
\newcommand*{\scriptT}{{\mathcal{T}}}
\newcommand*{\scriptU}{{\mathcal{U}}}
\newcommand*{\scriptX}{{\mathcal{X}}}
\newcommand*{\Cnk}[2]{\left(\begin{matrix}#1\\#2\end{matrix}\right)}
\newcommand*{\im}{{\mathbf i}}
\newcommand*{\id}{{\mathrm{id}}}
\newcommand*{\compl}{^\complement}
\newcommand*{\dotprod}[2]{{\left\langle{#1},{#2}\right\rangle}}
\newcommand\matr[1]{\left(\begin{matrix}#1\end{matrix}\right)}
\newcommand\matrd[1]{\left|\begin{matrix}#1\end{matrix}\right|}
\newcommand\arr[2]{\left(\begin{array}{#1}#2\end{array}\right)}

\DeclareMathOperator{\divby}{\scalebox{1}[.65]{\vdots}}
\DeclareMathOperator{\toto}{\rightrightarrows}
\DeclareMathOperator{\ntoto}{\not\rightrightarrows}

\newcommand{\undercolorblack}[2]{{\color{#1}\underline{\color{black}#2}}}
\newcommand{\undercolor}[2]{{\colorlet{tmp}{.}\color{#1}\underline{\color{tmp}#2}}}

\usepackage{adjustbox}

\geometry{margin=1in}
\usepackage{fancyhdr}
\pagestyle{fancy}
\fancyfoot[L]{}
\fancyfoot[C]{Иванов Тимофей}
\fancyfoot[R]{\pagename\ \thepage}
\fancyhead[L]{}
\fancyhead[R]{\leftmark}
\renewcommand{\sectionmark}[1]{\markboth{#1}{}}

\setcounter{tocdepth}{5}
\usepackage{amsthm}
\usepackage{chngcntr}

\theoremstyle{definition}
\newtheorem{definition}{Определение}
\counterwithin*{definition}{section}

\theoremstyle{plain}
\newtheorem{theorem}{Теорема}
\counterwithin*{theorem}{section} % Without changing appearance
\newtheorem{lemma}{Лемма}
\counterwithin*{lemma}{section}
\newtheorem{corollary}{Следствие}[theorem]
\counterwithin{corollary}{theorem} % Changing appearance
\counterwithin{corollary}{lemma}
\newtheorem*{claim}{Утверждение}
\newtheorem{property}{Свойство}[definition]

\theoremstyle{remark}
\newtheorem*{remark}{Замечание}
\newtheorem*{example}{Пример}


%\renewcommand\qedsymbol{$\blacksquare$}

\counterwithin{equation}{section}


\fancyhead[L]{Методы оптимизации}

\begin{document}
    \tableofcontents
    \section{Введение.}
    Что вообще такое оптимизация? Поиск некоторого оптимального значения. Обычно это задачи на тему максимизации/минимизации значения некоторой функции (и поиска аргумента, при котором оно достигается).\\
    Встречается это дело практически везде. В машинном обучении мы ищем параметр модели, когда она лучшим образом что-то предсказывает. Или мы хотим минимизировать невязку в системе линейных уравнений. Или мы проектируем какое-то устройство, и нам нужно сделать так, чтобы оно работало оптимально. Отсюда есть много слабо связанных методов и разделов, которые созданы для разных задач.\\
    О'кей, вот хотим мы найти
    $$
    \operatorname*{argmin}_{x\in X}f(x)
    $$
    Тут бывают разные $X$ и разные $f$. Самый большой раздел~--- когда $X$ это $\mathbb R^n$. Тут существует много разных задач (тот же поиск коэффициентов нейросети) и много методов. Методы делятся на 0, 1 и 2 порядка в зависимости от гладкости $f$.
    \begin{enumerate}[1.]
        \addtocounter{enumi}{-1}
        \item Нулевого порядка (differentiation-free) не опираются на гладкость никак и не используют производную (например, покоординатный спуск, симплекс-метод). При этом тут всё равно обычно полагают непрерывность $f$, иначе жить совсем грустно.
        \item Методы первого порядка используют гардиент. Самый известный метод~--- градиентный спуск, когда мы идём в направлении антиградиента. И отсюда метод стохастического градиентного спуска (когда мы не вычисляем градиент точно) и некоторые другие вариации. И вдобавок бывают методы сопряжённых градиентов, сопряжённых направлений, сопряжённых невязок. Многие, кстати, требуют от функции ещё каких-то условий.
        \item Методы 2 порядка полагают, что функция дифференцируется два раза и либо явно вычисляют вторые производные (тогда это какие-то ньютоновские методы), либо используют просто их наличие, а явно не вычисляют (квазиньютоновские методы).
    \end{enumerate}
    Вдобавок к этому есть какие-нибудь методы, основанные на случайности (метод Монте-Карло, например). И есть ещё методы, которые в одномерном случае работают (и многие многомерные методы имеют одномерные методы как подзадачу). Что у нас есть в одномерном случае?
    \begin{itemize}
        \item Дихотомия. Если предполагается, что функция имеет один минимум, то можно пилить отрезок пополам и искать. Или не пополам, а в отношении золотого сечения или ещё как-то.
        \item Полиномиальные методы. Можно считать, что наша функция близка к многочлены и искать минимумы этого многочлена.
    \end{itemize}
    Всё вышеперечисленное можно сочетать: сначала построить приближение многочленом, если там какой-то кринж, то воспользоваться другом методом.\\
    Ещё стоит заметить, что у нас есть локальные минимумы, и иногда нам хватит локального, а не глобального. И методы у нас обычно ищут локальный метод, потому что искать глобальный минимум сложно и очень немногие методы могут это сделать (и ещё и от функции чего-то требуют).\\
    Ещё есть отдельный подкласс задач, когда $X=\mathbb N$ или $X=\mathbb Z$ или большое конечное множество илии вообще какие-то не-числа (обычно дискретной природы). Такого сорта задачи называются задачами целочисленной оптимизации. Конкретно с целыми числами~--- задачами целочисленного программирования. Тут методы совершенно особые, и мы их в этом курсе затрагивать не будем.\\
    Бывают более сложные варианты, когда у нас $X=\mathbb R\times\mathbb N$, например. Это задачи смешанного программирования, и с ними вообще всё плохо жить.\\
    И ещё есть задачи, когда $X$~--- это некоторое подмножество $\mathbb R^n$. Такие задачи~--- задачи с ограничениями. К ним, кстати, обычно применяется другая нотация:
    $$
    \operatorname*{argmin}_{\substack{x\in\mathbb R^n\\c_i(x)=0\\c_j(x)\geqslant0}}f(x)
    $$
    То есть обычно ограничения~--- это равенство или неравенство. Это тоже достаточно большой и сложный раздел со своими методами. Иногда задача сводится к задаче без ограничений, иногда приходится решать как есть своими особыми методами. Например, есть линейное программирование, когда у нас
    $$
    f(x)=c_0+c_1x_1+c_2x_2\cdots+c_nx_n\qquad A_1x=b_1,A_2x\geqslant b_2
    $$
    Это много где применимая задача, к ней много что сводится. А решается она, например, симплекс-методом.\\
    Примером задачи, которая сводится к задаче линейного программирования,~--- транспортная задача. У нас есть какие-то фабрики и пункты продажи товара. Возникает вопрос с какой фабрики в каком количестве что куда вести. Каждая фабрика производит фиксированное число в единицу времени, у каждого магазина есть спрос (и мы не хотим создавать дефицит) и есть стоимость доставки на единицу товара из каждой фабрики в каждый магазин.\\
    Ещё есть задачи квадратичного программирования, когда
    $$
    f(x)=x^TAx\qquad A>0
    $$
    Положительная определённость матрицы хороша тем, что минимум у этой функции будет единственным.\\
    Ещё выпуклое программирование бывает, когда $f$ выпуклая функция и следовательно имеет единственный минимум.
    \section{Решение задачи без ограничений.}
    Тут есть два класса методов~--- линейный поиск и доверительные регионы. Что интересно, доверительные регионы показывают себя лучше (но и пишутся сложнее).
    \begin{itemize}
        \item Линейный поиск каким-то берёт точку и исходя из каких-то факторов (тот же антиградиент, например) идём от этой точки в определённом направлении. На какой шаг идём~--- тоже зависит от метода. Например, это может быть какой-то константный шаг (как в обычном градиентном спуске). Но это может быть плохо. При слишком большом шаге мы сильно проиграем, а если и не так, то, может, нам придётся долго ждать. Кстати, да, обычно стоит цель не только найти минимум, но и сделать это быстро. <<Быстро>>, кстати, измеряется обычно не в секундах, а к количестве раз, сколько мы вычислим функцию и/или её градиент.
        \item В случае доверительного региона мы строим некоторую модель для нашей функции, ищем минимум модели, уточняет модель, повторяем. Что такое модель? Какая-то функция. Самая простая~--- квадратичная (в некоторой окрестности локального минимума функция достаточно близка к квадратичной). При этом, чем лучше наша функция соотвествует модели, тем больше мы можем расширить область для следующей модели.
    \end{itemize}
    В доверительные регионы также входит задача нахождения гиперпараметров. У нас есть какие-то параметры модели (шаг спуска, например). И тут внутри нашей задачи оптимизации возникает ещё одна задача оптимизации, и тут всё совсем грустно, потому что не хочется очень много раз вычислять функцию. И тут обычно берут несколько комбинаций параметров и либо равномерно раскидывают по пространству параметров, либо случайно (второе лучше). Но ещё есть баесовский метод, когда мы итеративно уточняем модель по схеме, схожей с методом доверительных регионов.\\
    Ну и небольшой офф-топ с просто интересным алгоритмом (генетическим алгоритмом), который выглядит так: берём несколько случайных точек (интерпретируем их как популяцию), выбираем несколько лучше, особым образом скрещиваем их, добавляем немного случайности (мутации) и так создаём новое поколение. И этот алгоритм тоже довольно хорошо работает для довольно многих задач (и, кстати, не требует дифференцирования). Более того, он может работать даже не на числах, а хоть даже на строках.
    \subsection{Задача неограниченной локальной минимизации}
    То есть есть некоторая функция $f\colon\mathbb R^n\to\mathbb R$, и мы хотим найти, на каких аргументах достигается минимум. Минимум мы ищем локальный, потому что глобальный искать очень трудно.\\
    Для начала вспомним вообще, что такое минимум, какие у него есть свойства и т.д. Для начала вспомним формулу Тейлора, а точнее несколько следствий.
    $$
    f(x+p)=f(x)+\nabla f(x+tp)^Tp\qquad t\in (0;1)
    $$
    Это работает для гладкой функции $f$, и вообще дальше предполагаем, что функция у нас гладкая необходимое количество раз.
    $$
    f(x+p)=f(x)+\nabla f(x)^Tp+\frac12p^T\nabla^2f(x+tp)p\qquad t\in(0;1)
    $$
    Здесь и далее $\nabla$~--- градиент, а $\nabla^2$~--- гессиан (матрица вторых производных).\\
    Так вот, что же такое у нас минимум. Ну, с определением понятно, а мы хотим свойства и признаки.
    \begin{itemize}
        \item Если $x^*$~--- точка минимума $f$, то $\nabla f(x^*)=0$.
        \item Если $x^*$~--- точка минимума $f$, то $\nabla^2 f(x^*)\geqslant0$.
        \item Если $\nabla^2 f(x^*)>0$, то $x^*$~--- точка минимума $f$.
    \end{itemize}
    Также есть такое весёлое свойство как выпуклость (там искать минимум вообще очень просто, и локальный минимум будет глобальным), но это редкое свойство.\\
    Методов у нас очень много, нет одного правильного, разные методы лучше в разных случаях,.. Стоп. Что такое <<лучше>>? Смотрите. Обычно мы решаем всё численным методом, то есть строим последовательность, которая сходится к минимуму. Бесконечно строить эту последовательность мы не можем, поэтому надо остановиться. И мы получим некоторую точность, а для достижения этой точности придётся некоторое количество раз вычислить функцию, её градиент и её гессиан. И вот эти три количества и являются отражением эффективности метода.\\
    Большую роль в этом всём играет размерность пространства параметров. В случае, если $n=10$, нам хорошо подойдут одни методы, а если $n=100000$, то другие (например, считать для такой размерности в явном виде гессиан очень трудно; и гессиан надо либо приближённо считать, либо не использовать вообще). В современных нейронных сетях десятки и сотни тысяч параметров, где уже дорого использовать не только гессиан, но и градиент.
    \paragraph{Градиентный спуск.}
    Что же такое градиентный спуск? Метод выглядит так. Возьмём $x_0$~--- некое начальное приближение. Как его выбрать~--- тоже задача непростая, в зависимости от него мы получим разные результаты. Если мы примерно догадываемся, где минимум, можно попытаться выбрать точку близко к нему, например (хотя это не гарантирует нам быструю сходимость). Впрочем сейчас не об этом.\\
    Так вот, у нас есть начальное условие, и мы двигаем наши параметры по формуле
    $$
    x_{k+1}=x_k+\alpha p_k
    $$
    При этом хороший вопрос~--- когда остановится. Есть разные соображения о том, когда останавливаться. Можно смотреть на условия минимума и как-то их проверять. Можно смотреть на Гессиан, если нам нужна высокая гарантия, что мы нашли минимум, но это делают достаточно редко, потому что это сложно. Но в любом случае это не делают единственным критерием. Вторым критерием делают изменение $x$. Если он меняется очень мало, с большой вероятностью мы в тупике.\\
    Ещё нам часто хочется, чтобы функция убывала с ростом $k$ (это, опять же, не что-то универсальное, в стохастическом градиентном спуске это не выполняется, но тут хочется). Для этого надо наложить условия на $p_k$:
    $$
    p_k^T\nabla f(x_k)<0
    $$
    Самое очевидное~--- в качестве $p_k$ взять $-\nabla f(x_k)$. Это, собственно, и есть градиентный спуск. Это не единственный вариант, что можно делать, но один из. Альтернативные пути: можно смотреть на предыдущие шаги каким-то образом (метод сопряжённых градиентов, например) или можно пользоваться Ньютоновскими шагами.\\
    Вопрос: что такое $\alpha$ (в машинном обучении называют коэффициент обучения)? Простейший вариант~--- какая-то константа. Но также можно менять $\alpha$ в зависимости от номера шага (называется learning rate scheduling). Впрочем, такое не очень имеет смысл использовать в обычном градиентном спуске, помогает только в стохастическом.\\
    Что ещё можно делать? Можно искать минимум вдоль направления. Тут в первую очередь возникают два вопроса: как и насколько это нам надо (типа, насколько это нужно, если это просто один из шагов алгоритма). На второй мы не ответим, а про первый немного поговорим. Для начала мы ищем интервал, где находится минимум. Как? Посмотрим на условие $p_k^T\nabla f(x_k)<0$. Это даёт направление, в котором функция убывает. Мы смотрим на направление, куда функция убывает, и идём туда с каким-то шагом до тех пор, пока функция убывает.  Скорее всего наша функция не убывает бесконечно, а значит когда начнёт возрастать, мы найдём три точки, где центральная меньше боковых, то есть наши отрезок, где лежит минимум (bracketing называется). Дальше можно использовать деление пополам (или какие-то его вариации типа золотого сечения или ещё чего). Можно взять точки, провести многочлен через них и взять минимум (это интерполяция). Можно совместить интерполяцию с дихотомией. Или можно использовать что-то, что использует производные.
\end{document}